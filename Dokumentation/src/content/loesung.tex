\chapter{Ausgewählte Lösung}
%TODO überarbeiten issue #18

Neben den Beiden anderen Projekten haben wir uns für das Projekt von Ullrich
Radig Entschieden. Die Vorteile des Projektes gegenüber Ethersex oder Elektronik
2000 liegen darin, das die beiden Projekte zu speziell und umfangreich für unsere
Anforderungen sind. Zum einen war es unsere Aufgabe eine möglichst umfangreiche
Website zu erstelle, die Struckur von Ethersex ist für eine auf die Website
fokusierte Programmierung schlicht zu umfangreich und schlechter anpassbar.

\section{Der Webserver}
%TODO überarbeiten issue #18

Als Basis für unser Projekt haben wir die Firmware von Ulrich Radig verwendet.
Zusätzlich von der Uhrsprungsversion von Ulrich Radig gibt es noch eine Etwas
vereinfachte Version von Günther Menke. Wir haben wir uns für die vereinfachte
Variante von Günther Menke entschieden. Die Unterschiede zwischen beiden
Versionen belaufen sich auf das entfernte Kamera-Feature und um
zusätzlichen Quellcode für einen alternativen Netzwerkcontroller.

\subsection{Änderungen}

Zu der bereits vereinfachten Version von Günther Menke mussten wir für unser
Projekt noch Funktionalität von der ursprünglichen Version entfernen. Das
Problem lag darin, das wir für die Dateien der Website möglichst viel freien
Speicherplatz benötigen, der von den entsprechenden Funktionen belegt wurde.
Schlussendlich wurde Folgende Funktionalität aus der Version von Günther Menke
entfernt:

\begin{itemize}
  \item \textbf{(WOL) Wake on Lan} Funktionalität um andere Geräte im Netzwerk
  durch bestimmte Datenpackete aufzuwecken.
  \item \textbf{Sendmail} Senden von E-Mails.
  \item \textbf{Weather} Ermitteln von Wetterdaten.
  \item \textbf{(NTP) Network Time Protocol} Empfangen von Internetzeit
  \item \textbf{(DNS) Domain Name System} Beantwortung von Anfragen zur
  Namensauflösung.
  \item \textbf{(USART) Universal Synchronous and Asynchronous Serial Receiver and
  Transmitter} eine Schnittstelle im Mikrocontroller zum Daten
  Austausch mit PC über die COM-Schnittstelle.
  \item \textbf{(Telnet) Telecommunication Network} zeichenorientierten
  Datenaustausch über eine TCP-Verbindung.
  \item \textbf{(CMD) Command Control} Verwaltung der Telnet Konsolen Befehle.
\end{itemize}

\subsection{Einbindung der Website}

Die Website, welche hauptsächlich aus verschiedenen .html und .js Dateien
besteht, ist mangels Dateisystem für unsere Firmware nicht verwendbar. Die
gesamten Dateien müssen in einer C-Headerdatei gebunden werden.
Das erstellen der Headerdatei erfolgt über das beim Projekt beigelegte
\ac{HHC} Werkzeug. Eine Beispiel zum erstellen der \textrm{webpage.h} 
Datei und eine Erklärung des \ac{HHC} gibt es im Kapitel Werkzeuge
\ref{chap:hintergrund.HHC}. Eine Anleitung zur Ausführung des \ac{HHC} gibt es im
Benutzerhandbuch \ref{chap:benutzerhandbuch.HHC}.
Abschließend ist noch zu erwähnen, das die \textrm{webpage.h} nicht für manuelle
Bearbeitung gedacht ist. Dies geschieht ausschließlich über die Quell-Dateien
und anschließendem umwandeln mit dem \ac{HHC}.

\section{Die Website}

Der Aufbau der Webseite ist in mehrer Dateien aufgeteilt. So haben wir mehrer
.js und .css Dateien. In der index.html wird alles nur zusammengetragen.
\newline


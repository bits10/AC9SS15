%Korrekturgelesen: Ann-Sophie Dietrich
\chapter{Ausgewählte Lösung}

Neben den beiden anderen Projekten haben wir uns für das Projekt von Ulrich
Radig entschieden. Die Vorteile des Projektes gegenüber Ethersex oder Elektronik
2000 liegen darin, das die beiden Projekte zu speziell und umfangreich für unsere
Anforderungen sind. Da es zu unserem Aufgabengebiet gehörte eine für den
Benutzer möglichst umfangreiche Website zu erstellen kam uns die gut anpassbare
Lösung entgegen.

\section{Der Webserver}

Als Basis für unser Projekt haben wir die Firmware von Ulrich Radig verwendet.
Zusätzlich von der Ursprungsversion von Ulrich Radig gibt es noch eine etwas
vereinfachte Version von Günther Menke. Wir haben wir uns für die vereinfachte
Variante entschieden. Die Unterschiede zwischen beiden
Versionen belaufen sich auf das entfernte Kamera-Feature und auf
zusätzlichen Quellcode für einen alternativen Netzwerkcontroller.

\subsection{Änderungen}

Obwohl wir die bereits im Funktionsumfang vereinfachte Version
von Günther Menke verwenden, gab es trotzdem einiges an Funktionalität, welche wir
aus der ursprünglichen Version entfernt haben.
Das Problem lag darin, dass wir zum einen für die Dateien der Website möglichst
viel freien Speicherplatz benötigen, der von den entsprechenden Funktionen
belegt wurde. Zum anderen, dass die entfernten Funktionen nicht für unser
Projekt benötigt wurden. Schlussendlich wurden folgende Funktionalitäten aus der Version
von Günther Menke entfernt:

\begin{itemize}
  \item \textbf{(WOL) Wake on Lan} Funktionalität um andere Geräte im Netzwerk
  durch bestimmte Datenpakete aufzuwecken.
  \item \textbf{Sendmail} Senden von E-Mails
  \item \textbf{Weather} Ermitteln von Wetterdaten
  \item \textbf{(NTP) Network Time Protocol} Empfangen von Internetzeit
  \item \textbf{(DNS) Domain Name System} Beantwortung von Anfragen zur
  Namensauflösung
  \item \textbf{(USART) Universal Synchronous and Asynchronous Serial Receiver and
  Transmitter} eine Schnittstelle im Mikrocontroller zum Datenaustausch mit PC über die COM-Schnittstelle
  \item \textbf{(Telnet) Telecommunication Network} zeichenorientierter
  Datenaustausch über eine TCP-Verbindung.
  \item \textbf{(CMD) Command Control} Verwaltung der Telnet-Konsolenbefehle.
\end{itemize}

\subsection{Einbindung der Website}

Die Website, welche hauptsächlich aus verschiedenen .html und .js Dateien
besteht, ist mangels Dateisystem für unsere Firmware nicht verwendbar. Die
gesamten Dateien müssen in einer C-Headerdatei gebunden werden.
Das Erstellen der Headerdatei erfolgt über das beim Projekt beigelegte
\ac{HHC} Werkzeug. Eine Beispiel zum Erstellen der \textrm{webpage.h} 
Datei und eine Erklärung des \ac{HHC} gibt es im Kapitel Werkzeuge
\ref{chap:hintergrund.HHC}. Eine Anleitung zur Ausführung des \ac{HHC} gibt es im
Benutzerhandbuch \ref{chap:benutzerhandbuch.HHC}.
Abschließend ist noch zu erwähnen, dass die \textrm{webpage.h} nicht für manuelle
Bearbeitung gedacht ist. Dies geschieht ausschließlich über die Quell-Dateien
und anschließendem umwandeln mit dem \ac{HHC}.

\section{Die Website}

Der Aufbau der Webseite ist in mehrere Dateien aufgeteilt. So haben wir mehrere
.js und .css Dateien. In der index.html wird alles nur zusammengetragen.

\subsection{HTML und CSS}
Es gibt nur eine html Seite in der alle anderen Dateien aufgerufen werden und zur
Hauptseite wird: Die index.html\newline
Die CSS-Elemnte sind in mehrere Dateien aufgesplittet. So kann die input.css
gegen eine andere input.css ausgetauscht werden. Die input.css ist die Datei für
das Desgin mit den Farben, Formen, usw.\newline
Eine weitere .css Datei sollte nicht ausgetauscht werden (board.css).

\subsubsection{board.css}
Diese Datei behandelt alle Definitionen über die Platine in dem 'Status
Tab' der Webseite. Alle Defenitionen über die Platine sind direkt in der
index.html in einer komplexen div-Verschaltung eingebunden. So können einzelne
div-Elemnte als klickbar und verlinkbar zu anderen Elemnten sein. Eine
veränderung der Verschachtelung oder im CSS-Code kann zu Anzeigefehler oder ein
NichtAnzeigen der Webseite führen.\newline
Alle Div-Elemnte sind abgemessen und auf die passende Größe ausgerichtet. 

\subsection{JavaScript}
\subsubsection{Generell}
In \textrm{rest.js} und \textrm{favorite.js} werden die einzelnen Pins über eine
ID unterschieden. Diese ID wird vor allem als Parameter für diverse Funktionen
wie \textrm{getValu(id)} verwendet. Die ID setzt sich immer aus dem Buchstaben des
Ports, sowie aus der Nummer des Pins zusammen. Mögliche IDs sind folglich [A0
bis A7], [C0 bis C7] und [D0 bis D7].

\subsubsection{rest.js}
Die \textrm{rest.js} beinhaltet den gesamten Code zum Abrufen von Daten beim
Server.

Hierbei bieten die Funktionen \textrm{loadUrl(\ldots)} und
\textrm{loadUrlAsync(\ldots)} den Kern der Serverkommunikation. Mit ihnen können
beliebige URLs geladen werden, \textrm{loadUrlAsync(\ldots)} bietet zusätzlich
die Möglichkeit POST-Parameter zu übergeben. Der grundlegende Unterschied
zwischen diesen beiden Methoden ist, dass \textrm{loadUrl(\ldots)} den
Programmablauf blockiert, bis die Daten vollständig geladen sind, wohingegen
\textrm{loadUrlAsyn(\ldots)} die Daten parallel im Hintergrund lädt. Sobald alle
Daten erfolgreich geladen sind, wird die Funktion aufgerufen, die
\textrm{loadUrlAsync(\ldots)} als Parameter übergeben wurde. Weitere
Informationen, vor allem zur Parametisierung der Funktionen, sind der
Sourcecode-Dokumentation zu entnehmen.\\
\\
In der Funktion \textrm{initRest()} wird zum ersten Mal die Funktion
\textrm{startNewRefreshTask()} aufgerufen. Diese ruft die Funktion
\textrm{refreshValues()} auf, welche sich selbst rekursiv immer wieder nach
einer gewissen Pause (der Polling-Frequenz, welche in den Einstellungen
festgelegt werden kann) aufruft. Die Polling-Frequenz wird in dem Attribut
\textrm{pollingFreq} gespeichert. Der so erzeugte Refersh-Task, der
vollautomatisch die Werte aktualisiert, kann über \textrm{cancelRefreshTask()}
wieder gestoppt werden. Die Werte werden danach natürlich nicht mehr
aktualisiert, bis mit \textrm{startNewRefreshTask()} ein neuer Task gestartet
wird. Nach jedem Aktualisieren der Werte wird die Funktion aufgerufen, die in
dem Attribut \textrm{onValuesChanged} gespeichert wird. In unserer
Implementierung ist diese Funkion in der \textrm{ui.js} implementiert und
organisiert die Aktualisierung der Oberfläche, damit die aktuellen Werte
dargestellt werden. Der Wert des Attributes \textrm{onValuesChanged} kann mit
der Funktion \textrm{setOnValuesChanged(\ldots)} manipuliert werden. Die zuletzt
empfangenen Werte werden in \textrm{cachedValues} gespeichert und sind über
diverse Getter wie \textrm{getValue(id)} verfügbar.

Das Stoppen eines bestehenden Tasks wird realisiert, indem das Attribut
\textrm{taskIdCounter} inkrementiert wird. \textrm{refreshValues()} überprüft am
Anfang, ob der aktuelle Wert von \textrm{taskIdCounter} mit dem Wert
übereinstimmt, mit der der aktuell ausgeführte Task gestartet wurde. Ist dies
nicht der Fall, weil der Task beendet oder ein neuer gestartet wurde, beendet
sich der Task automatisch selbst, indem die Funktion \textrm{refreshValues()}
nicht erneut rekursiv aufgerufen wird.\\
\\
In der Funktion \textrm{initRest()} werden einmalig die REST-URLs
\textrm{/rest/info} und \textrm{/rest/pininfo} geladen. Diese beinhalten nur
statische Informationen, welche sich nie ändern. Die empfangenen JSON-Strukturen
werden in ein JavaScript Objekt geparst und in \textrm{cachedInfo} und
\textrm{cachedPinInfo} gespeichert. Der Inhalt dieser Objekte wird über
die Getter \textrm{getInfo()} und \textrm{getPinInfo()} zur Verfügung gestellt.
Für \textrm{cachedPinInfo} sind zudem Getter wie
\textrm{getDefaultDescription(id))} vorhanden.


\subsubsection{favorites.js}

In der \textrm{favorites.js} wird eine Liste aller Pins angelegt. \\
Beim Hinzufügen eines Pins wird über \textrm{addPin(id)} überprüft, ob der Pin bereits in der Liste existiert. Ist dies nicht der 
Fall, so wird ein neuer Pin mit Standardwerten in die \textrm{favoritelist} geschrieben. 
Der Nutzer kann für jeden Pin eine
Beschreibung, sowie eine Funktion und ob dieser Pin als Favorit markiert werden soll, 
hinterlegen. Dies geschieht mit den Funktionen \textrm{setDescription(id, des)}, 
\textrm{setFunction(id, func)} und \textrm{toggleFavorite(id)}.\\
Bei jeglichen Gettern(z.B. \textrm{getFunction()}, \textrm{getDescription()})von Pin wird ebenfalls überprüft, 
ob der Pin bereits in der Liste existiert. Existiert er, werden die 
geforderten Werte zurückgegeben. Existiert er nicht, wird er mit Standardwerten in die \textrm{favoritelist} geschrieben.\\
 Diese Liste wird über die JavaScript-Datei 
\textrm{db.js}, abgespeichert und bei Neustart aus dieser ausgelesen. Diese \textrm{save()} Funktion übergibt die 
\textrm{favoritelist} mithilfe von \textrm{JSON.stringify()}.\\
Zusätlich kann diese durch Import und Export Funktionen als reiner Fließtext ausgegeben und auf anderen Rechnern 
eingefügt werden, damit die Daten dort ebenfalls exisiteren.\\
Ursprünglich sollte \textrm{favorites.js} nur eine Liste der als Favorit markierte Pins speichern, 
was jedoch nach einigen Überlegungen zu einer Liste aller Pins umgeändert wurde. Der Name 
\textrm{favorite.js} wurde jedoch beibehalten, sowie die Funktion, Pins als Favorite zu setzen. 
Dies geschieht über ein Flag, welches \textrm{true} oder \textrm{false} sein kann.\\

\subsubsection{db.js}
Die \textrm{db.js} besteht aus 2 Funktionen: 

\textrm{getDb(key, defaultValue)} liefert den gespeicherten Wert für den
gegebenen \textrm{key}, welcher immer eine Zahl oder ein String sein sollte. Ist
kein Wert mit dem gegebenen Key vorhanden, wird das
übergebene \textrm{defaultValue} zurückgegeben.

\textrm{putDb(key, value)} speichert das gegebenen \textrm{value} unter dem
gegebenen \textrm{key} ab. Wird jetzt \textrm{getDb(key, defaultValue)} mit dem
gleichen \textrm{key} aufgerufen, wird der zuvor abgespeicherte Wert
zurückgegeben.

Alle Mit \textrm{db.js} gespeicherten Werte sind auch nach einem Neuladen der
Seite weiterhin verfügbar, solange nicht die IP-Adresse des Pollin Net-IO Boards
verändert wird. Die verwendeten Keys müssen natürlich eindeutig sein.

\subsubsection{ui.js}
Die \textrm{ui.js} beinhaltet den gesamten Code für das dynamische Gestalten der
Webseite\newline
Ein der wichtigsten Funktionen, die generell verwendet wird, ist die
\textrm{getEl(id)}.
Diese Funktion verkürzt die Standardfunktion von JavaScript
\textrm{getElementById(id)}, um den Sourcecode zu kürzer zu halten,
um den daraus resultierenden Speicherbedarf der Webseite zu
minimieren.\newline

\textrm{initUi()} sollte nur einmal beim Seitenaufruf geladen werden und
initialisiert die Bneutzerschnittstellen. \textrm{setMain(div)} setzt die
Hauptkoponenten von dem Frame 'div'.\\
\textrm{addClass(clazz, item)} fügt eine CSS Klasse zu dem gegebenen Item hinzu,
während die \textrm{removeClass(clazz, item)} die CSS Klasse von dem Item wieder
löscht.\\
\textrm{getHighlightElem(id)} makiert die digitalen Pins, oder makiert die
Checkboxen der analogen Pins.
Folgende Funtkionen generieren die dynamische Sidebar: \textrm{setSidebarId(id)}
listet alle Informationen über die Pins in der Sidebar auf. Die
\textrm{updateSidebarValues()} hält die Werte der einzelnen Pins auf den
aktuellsten Stand.\\
Die Favoriten Tabelle wird mittels der Funktion \textrm{updateFavoritesTable()}
erneuert und mit \textrm{updateFavoritesTableValues()} werden die Pininformationen 
der Favoriten auf den aktuellsten Stand gebracht. Ändern sich die
Favoriten wird \textrm{onFavoritesChanged()} aufgerufen.\\
\textrm{text}\\
Das seperat aufpoppende Fenster um die Pins / Ports besser zu konfigurieren wird
mit \textrm{startConfigurePin(id)}. Beim Aufruf wird die ID des Pins benötigt.
In der Funktion werden die restlichen Informationen zu dem Pin abegrufen und
eingetragen. In dem Fenster sind die Buttons 'OK' und 'zurücksetzen', die dann
beim bestätigen \textrm{endConfigurePin(reset)} aufruft. reset ist der Flag
zum zurücksetzen des Pins. Ist er auf true, wird der Pin zurückgesetzt, sollte
er auf false sein, werden die neuen Konfigurationen zum Pin abgespeichert.\\
Um die index.html nicht noch mit weiteren vielen div-Elemnten zu
überfüllen, werden die gleichen div-Elemente mit \textrm{write(count, string,
ids)} generiert. Diese Funktion wird in folgenden verwendet:
\textrm{writePinCheck(ids)}, \textrm {writePinCheckMinus(count)},
\textrm{writePinCheckPlus(count)}, \textrm{writePinCheckNone(count)},
\textrm{writePinMinusBox(count)}, \textrm{writePinCheckAnalog(ids)} und
\textrm{writeAnalogBox(ids)}.




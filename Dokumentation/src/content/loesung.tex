\chapter{Ausgewählte Lösung}

\section{Der Webserver}

Als Basis für unser Projekt haben wir die Firmware von Ulrich Radig verwendet.
Zusätzlich von der Uhrsprungsversion von Ulrich Radig gibt es noch eine Etwas
vereinfachte Version von Günther Menke. Wir haben wir uns für die vereinfachte
Variante von Günther Menke entschieden. Die Unterschiede zwischen beiden
Versionen belaufen sich auf das entfernte Kamera-Feature und um
zusätzlichen Quellcode für einen alternativen Netzwerkcontroller.

\subsection{Änderungen}

Zu der bereits vereinfachten Version von Günther Menke mussten wir für unser
Projekt noch Funktionalität von der ursprünglichen Version entfernen. Das
Problem lag darin, das wir für die Dateien der Website möglichst viel freien
Speicherplatz benötigen, der von den entsprechenden Funktionen belegt wurde.
Schlussendlich wurde Folgende Funktionalität aus der Version von Günther Menke
entfernt:

\begin{itemize}
  \item \textbf{(WOL) Wake on Lan} Funktionalität um andere Geräte im Netzwerk
  durch bestimmte Datenpackete aufzuwecken.
  \item \textbf{Sendmail} Senden von E-Mails.
  \item \textbf{Weather} Ermitteln von Wetterdaten.
  \item \textbf{(NTP) Network Time Protocol} Empfangen von Internetzeit
  \item \textbf{(DNS) Domain Name System} Beantwortung von Anfragen zur
  Namensauflösung.
  \item \textbf{(USART) Universal Synchronous and Asynchronous Serial Receiver and
  Transmitter} eine Schnittstelle im Mikrocontroller zum Daten
  Austausch mit PC über die COM-Schnittstelle.
  \item \textbf{(Telnet) Telecommunication Network} zeichenorientierten
  Datenaustausch über eine TCP-Verbindung.
  \item \textbf{(CMD) Command Control} Verwaltung der Telnet Konsolen Befehle.
\end{itemize}

\subsection{Einrichtung}

Die Einstellung des Webservers erfolgt über die \textrm{config.h} Datei. In der
\textrm{config.h} Datei, können Zum einen die verschiedenen Pins der Ports als Ein oder
Ausgang definiert werden. Dabei gibt es ein paar Eigenheiten zu beachten:
\begin{itemize}
  \item OUTA steht für den A Port, hier ist du beachten, das dieser Port die
  Analog zu Digital Wandler beherbergt. Mit aktiviertem Wandler ist es nicht
  möglich die Pins des Ports funktionierend als Ausgänge zu schalten, da die
  Spannung nicht gehalten wird.
  \item OUTB ist nur mit Vorsicht zu genießen. Hier handelt es sich um den Port
  der auf dem AVR-Net-IO für die Neztwerkkomunikation genutzt wird. Deswegen
  wird Port B auch nicht Standardmäßig definiert.
  \item OUTC dieser Port wird von Polin standardmäßig für die Ausgänge verwendet
  und ist von uns bereits so Modifiziert. Der gesamte Port wird auf dem
  AVR-NET-IO über den 25Pin D-Sub Stecker geleitet. Wenn der Fuse-Bit für JTag
  geschaltet ist werden 4 Pins des C Ports für das JTag Interface verwendet.
  \item OUTC Der C Port liegt auf dem AVR-Net-IO auf dem EXT Anschluss und ist
  für erweiterte Peripherie geplant, so kann hier ein Cardreader oder ein
  Erweiterungsboard angeschlossen werden.
\end{itemize}
Weiter kann die gewünschte IP-Adresse eingestellt werden, unter der das Gerät
erreicht werden kann. Wichtig ist hier, das kein anderes Gerät die selbe Adresse
im Netzwerk verwendet. Auch kann die Router IP-Adresse und Netzmaske angegeben
werden. Eine weitere Wichtige Einstellung ist die verwendete Mac Adresse
des Netzwerk Controllers. Diese wird über die Variablen MYMAC1-6 Definiert.

\subsection{Einbindung der Website}

Alle für den Betrieb der Website benötigten Dateien sind nicht über ein
Dateisystem vorhanden, sondern werden beim Zugriff des Benutzers auf den
Webserver über die \textrm{webpage.h} Datei geladen.
Das erstellen der .h erfolgt über das Beigelegte \textrm{HTML Header Compiler}
Werkzeug erstellt. Eine Beispiel zum erstellen der \textrm{webpage.h} Datei gibt
es im Tutorial Abschnitt. Das Werkzeug wird im Kapitel Werkzeuge detaillierter
erklärt. Abschließend ist noch zu erwähnen, das die \textrm{webpage.h} nicht für manuelle
Bearbeitung gedacht ist. Dies geschieht ausschließlich über die Quell-Dateien
und dem anschließenden umwandeln mit dem \textrm{HTML Header Compiler}.

\section{Die Website}

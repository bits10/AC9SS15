\chapter{Ausgewählte Lösung}
%TODO überarbeiten issue #18

Neben den Beiden anderen Projekten haben wir uns für das Projekt von Ullrich
Radig Entschieden. Die Vorteile des Projektes gegenüber Ethersex oder Elektronik
2000 liegen darin, das die beiden Projekte zu speziell und umfangreich für unsere
Anforderungen sind. Da es zu unserem Aufgabengebiet gehörte eine für den
Benutzer möglichst umfangreiche Website zu erstelle kam uns die gut anpassbare
lösung entgegen.

\section{Der Webserver}
%TODO überarbeiten issue #18

Als Basis für unser Projekt haben wir die Firmware von Ulrich Radig verwendet.
Zusätzlich von der Uhrsprungsversion von Ulrich Radig gibt es noch eine Etwas
vereinfachte Version von Günther Menke. Wir haben wir uns für die vereinfachte
Variante entschieden. Die Unterschiede zwischen beiden
Versionen belaufen sich auf das entfernte Kamera-Feature und um
zusätzlichen Quellcode für einen alternativen Netzwerkcontroller.

\subsection{Änderungen}

Obwohl wir die bereits die im Funktionsumfang vereinfachte Version
von Günther Menke verwendeten, gab es trotzdem einiges an Funktionalität die wir
aus der ursprünglichen Version entfernt haben.
Das Problem lag darin, das wir zum einen für die Dateien der Website möglichst
viel freien Speicherplatz benötigen, der von den entsprechenden Funktionen
belegt wurde. Zum anderen, das die entfernten Funktionen nicht für unser
Projekt benötigt wurden. Schlussendlich wurde Folgende Funktionalität aus der Version
von Günther Menke entfernt:

\begin{itemize}
  \item \textbf{(WOL) Wake on Lan} Funktionalität um andere Geräte im Netzwerk
  durch bestimmte Datenpackete aufzuwecken.
  \item \textbf{Sendmail} Senden von E-Mails.
  \item \textbf{Weather} Ermitteln von Wetterdaten.
  \item \textbf{(NTP) Network Time Protocol} Empfangen von Internetzeit
  \item \textbf{(DNS) Domain Name System} Beantwortung von Anfragen zur
  Namensauflösung.
  \item \textbf{(USART) Universal Synchronous and Asynchronous Serial Receiver and
  Transmitter} eine Schnittstelle im Mikrocontroller zum Daten
  Austausch mit PC über die COM-Schnittstelle.
  \item \textbf{(Telnet) Telecommunication Network} zeichenorientierten
  Datenaustausch über eine TCP-Verbindung.
  \item \textbf{(CMD) Command Control} Verwaltung der Telnet Konsolen Befehle.
\end{itemize}

\subsection{Einbindung der Website}

Die Website, welche hauptsächlich aus verschiedenen .html und .js Dateien
besteht, ist mangels Dateisystem für unsere Firmware nicht verwendbar. Die
gesamten Dateien müssen in einer C-Headerdatei gebunden werden.
Das erstellen der Headerdatei erfolgt über das beim Projekt beigelegte
\ac{HHC} Werkzeug. Eine Beispiel zum erstellen der \textrm{webpage.h} 
Datei und eine Erklärung des \ac{HHC} gibt es im Kapitel Werkzeuge
\ref{chap:hintergrund.HHC}. Eine Anleitung zur Ausführung des \ac{HHC} gibt es im
Benutzerhandbuch \ref{chap:benutzerhandbuch.HHC}.
Abschließend ist noch zu erwähnen, das die \textrm{webpage.h} nicht für manuelle
Bearbeitung gedacht ist. Dies geschieht ausschließlich über die Quell-Dateien
und anschließendem umwandeln mit dem \ac{HHC}.

\section{Die Website}

Der Aufbau der Webseite ist in mehrer Dateien aufgeteilt. So haben wir mehrer
.js und .css Dateien. In der index.html wird alles nur zusammengetragen.

\subsection{HTML und CSS}
Nur eine html Seite in der alle anderen Dateien aufgerufen werden und zur
Hauptseite wird: Die index.html\newline
Die CSS-Elemnte sind in mehrere Dateien aufgesplittet. So kann die input.css
gegen eine andere input.css ausgetauscht werden. Die input.css ist die Datei für
das Desgin mit den Farben, Formen, usw.\newline
Eine Weitere css Datei sollte nicht ausgetauscht werden (board.css).

\subsubsection{board.css}
Diese Datei behindelt alle Definitionen über die Platine in der Webseiten Status
Tab. Diese Definitionen werden in der index.html in einem 'div-Salat' zu einer
Platine, die auch klickbar ist, zusammengebaut. Löschen eines Divs verursacht
Darstellungsfehler, sowie ein nicht mehr Anzeigen der grafischen Platine.

\subsection{JavaScript}
\subsubsection{rest.js}
\subsubsection{favorites.js}
\subsubsection{db.js}
\subsubsection{ui.js}
Mithilfe dieser ui.js wird die Webseite dynamisch gestaltet. Hier sind die
grundlegenden Funktionen für den Webseiten Aufbau niedergeschrieben.\newline
Wichtige Funktionen, wie verkürzen von JavaScript Befehlen
(document.getElementByID()). Da dann nur noch die Kürzel verwenden werden,
verkürzt dies den Sourcecode und dies minimiert auch den Speicherbedarf der
Webseite.\newline


%Korrekturgelesen: Ann-Sophie Dietrich
\chapter{Einleitung}

Durch die immer weiter fortschreitende Vernetzung unserer alltäglichen Elektronik
wird das Verlangen nach netzwerkfähigen Mikrocontrollern
immer größer. \\
Wenn zum Beispiel der moderne Kühlschrank erkennen soll, wie gut er gerade
gefüllt ist oder die Heizung melden soll wie warm das Wasser ist, muss ein
entsprechend ausgestatteter Controller diese Informationen an den Nutzer
melden können, welcher dann entsprechend darauf reagieren oder über gespeicherte 
Funktionen den Controller entsprechend reagieren lassen kann.\\ 
Hier greift unser Projekt da mit dem Mikrocontroller und seinen
verschiedenen Ein- und Ausgängen unterschiedlichste Anwendungen ermöglicht
werden.

%Etwas kurz aber sonst gut
% Soll man das nochmal überarbeiten? Also länger machen? 

\section{Versionskontrolle}

Für die Versionskontrolle haben wir für unsere Projekt Dateien auf GitHub. Auf
die Funktionsweise von Git wird an dieser Stelle nicht weiter eingegangen,
weiterführende Hilfestellung gibt es auf der offiziellen Hilfe-Seite
\url{https://help.github.com/}. Die Projektseite kann unter der Adresse
\url{https://github.com/doofmars/Embedded-Webserver} erreicht werden. Um das
Repository zu "`clonen"' benötigt man folgende Adresse:
\url{https://github.com/doofmars/Embedded-Webserver.git}. Alternativ können die
Projektdaten unter \url{https://github.com/doofmars/Embedded-Webserver/archive/master.zip} 
komprimiert heruntergeladen werden.



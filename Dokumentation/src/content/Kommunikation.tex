\chapter{Kommunikation zwischen Server und Webseite}

\begin{itemize}
	\item Aufbau als REST-Schnittstelle
	\begin{itemize}
		\item Simpel implementierbar
		\item Technischer Standart
		\item Fertige Mechaniken in JavaScript
		\item Leicht erweiterbar um neue Funktionalitäten
	\end{itemize}
	\item Anforderungen an eine Lösung:
	\begin{itemize}
		\item Möglichst kompakt, wenig Bandbreite soll verbraucht werden
		\item Messwerte müssen besonders effizient übertragen werden
		\item Skalierbar - Egal ob ein oder 20 Ports verwendet werden
		\item Neben Abfragen der Messwerte auch manipulieren der IP, des DDR und der 
			  Ausgänge
		\item JSON-Encodiert, direkt in JavaScript Objekt überführbar
	\end{itemize}
\end{itemize}

%-----------------------------------------------------------------------------------------
\section*{Polling oder Push}

\begin{itemize}
	\item Daten können entweder per Polling vom Client (Webseite abgefragt werden) oder
		  vom Server bei einem Event (Änderung eines Eingangs) per Push geschickt werden	  
	\item Pro/Contra Push
	\begin{itemize}
		\item Pro: Daten werden nur übertragen wenn es wirklich nötig ist, kein unnötiger 
			  Datenverkehr
		\item Pro: Leistung bleibt auch mit mehreren Clients eher konstant 
			  (Genau Begründung ausarbeiten!)
		\item Pro: Board wird entlastet, kein DOS, mehrere Clients können Webseite 
		      trotzdem problemlos aufbauen 
	\end{itemize}
	\item Pro/Contra Polling
	\begin{itemize}
		\item Pro: Leichter zu implementieren - normale Dateiabfrage
		\item Contra: Es werden viele unnötige Daten übertragen
		\item Contra: Leistung nimmt mit steigender Anzahl von Clients ab
		\item Pro: Im Prinzip nicht langsamer als Push: Limitierende Komponente ist die 
			  Übertragungszeit! (Hier Bild von Chrome Netzwerkvehrkehr) 
	\end{itemize}
	
	\item Push bessere Lösung
	\item Umsetzung aber nicht möglich
	\item Server kann keine Verbindung zu Client aufbauen
	\begin{itemize}
		\item Javascript kann keinen Serverport öffnen
		\item Client hat keine öffentliche IP
	\end{itemize}
	\item Technische Lösung: HTML 5 Server Sent Events
	\begin{itemize}
		\item Technischer Standart, Eingeführt mit HTML 5
		\item Client ruft virtuelle URL auf, ähnlich REST
		\item So wird ein Stream geöffnet
		\item Bei einem Event schreibt der Server die Informationen in den Stream
		\item Client benutzt Javascript API um die jeweiligen Informationen zu lesen
	\end{itemize}
	\item Mit ATmega CPU und Radig Projrkt als Vorlage nicht lösbar
	\begin{itemize}
		\item Radig kann nur eine Verbindung handeln. Aufbau - Übertragung - Abbau für 
			  jede Datei
		\item HTTP 1.0
		\item Aufruf einer HTML5 Server Sent Event URL würde dazu führen das der Server 
			  blockiert ist
		\item Gesamte Struktur auf nur eine Verbindung ausgelegt
		\item Bei Umschreiben (sehr tiefer Eingriff!) stößt die CPU schnell an Grenzen
		\item Multitasking nötig, Stack für jede Verbindung führt zu RAM Problemen
		\item Sprengt vermutlich zeitlichen Rahmen
	\end{itemize}
	\item Daraus resutltiert die Verwendung von Polling
	\item Da Polling nicht langsamer ist, gibt es keine Nachteile bei Frequenzen von ca. 
		  200ms (auch bei zB. 3 Clients)
\end{itemize}

%-----------------------------------------------------------------------------------------
\section*{Aufbau der REST-Schnittstelle}

\begin{itemize}
	\item 6 URLS
	\item ..........Hier Zitat aus Präsentation, aber noch anpassen!
\end{itemize}
	
%-----------------------------------------------------------------------------------------
\section*{Implementierung der REST-Schnitstelle auf dem Server}

\begin{itemize}
	\item Benutzung der Struktur von Radig
	\item Platzhalter \%PortXY werden beim senden des Dokuments ersetzt durch passende 
		  Werte
	\item in /rest liegen Dateien values (mit Platzhaltern), pininfo und info (jeweils als
	      statisches Dokument)
	\item Dateien können von ausßen normal aufgerufen werden und werden ggfs. mit realen 
	      Informationen aufgefüllt
	\item Vorteil: Test außerhalb der Platine möglich, da Werte als String gekennzeichnet 
	      sind in JSON, Pins haben dann den Wert "\%PortXY" anstatt "0" oder "1"
	\item TODO: Wie werden die set URLs implementiert?
\end{itemize}

%-----------------------------------------------------------------------------------------
\section*{Implementierung der REST-Schnitstelle auf dem Client}

\begin{itemize}
	\item Aufruf der URLs im Hintergrund mit Ajax
	\item info und pininfo werden zu Beginn einmalig aufgerufen (synchron um zu 
	      gewährleisten das die Daten zur Verfügung stehen für andere Initialisierungen)
	\item values wird mit setTimeout(...) zyklisch assynchron aufgerufen
	\item synchrones aufrufen von values führt dazu, das sich die Webseite aufhängt, da
	      der (Single-)JavaScript Thread mit dem laden der Daten beschäftigt ist und nicht
	      für andere Aufgaben zur Verfügung steht. Bei einem assynchronen Aufruf werden 
	      die Daten von einem anderen Thread im Hintergrund geladen
	\item JSON-Text wird mit JSON.parse(...) in ein Objekt transformiert
	\item Informationen werden über entsprechende getter zur Verfügung gestellt
	\item Funktion onValueChanged wird jedes mal aufgerufen wenn neue Daten zur 
	      Verfügung stehen
	\item onError wird aufgerufen wenn ein Fehler in der Kommunikation aufgetreten ist
	\item Über setter werden die entsprechenden URLs assynchron aufgerufen und so die 
	      Daten an den Server übermittelt
\end{itemize}
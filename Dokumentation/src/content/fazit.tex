%Korrekturgelesen von : Ann-Sophie Dietrich
\chapter{Fazit}

Die Welt der Mikrocontroller steckt voller Möglichkeiten, jedoch enthält sie, gerade 
für Laien, noch einige Startschwierigkeiten.\\
Im Gegensatz zum Arbeiten mit Computern, bei welchen der Speicherplatz für 
einfache Programme schier unbegrenzt scheint, kommt es bei
den Mikrocontrollern mit ihrem geringen Speicher auf jedes Byte an. 
So bestand in unserem Projekt 
die Schwierigkeit nicht nur darin, den Server mit weiteren Funktionen auszustatten, sondern
auch bei der Programmierung möglichst auf Effizienz zu achten und den
bestehenden Webserver von nicht benötigten Funktionen zu befreien, sowie beim Hinzufügen 
neuer Funktionen streng auf den Speicherverbrauch zu achten.\\
Als eine weitere Herausforderung bei Mikrocontrollern kam noch die ganze elektronische
Seite hinzu. Als Informatiker haben wir durch das Studium kaum Berührung mit
diesem Thema gehabt und mussten uns vielerorts in die Themaktik einarbeiten.
Doch hat sich das Projekt im Endeffekt als handhabbarer erwiesen als anfangs gedacht. Die
Hauptaufgaben bestanden hier im Beschaffen von Bauteilen, dem
Erstellen von Platinen zum Testen der Funktionen oder dem Programmieren und
Debuggen des Mikrocontrollers.


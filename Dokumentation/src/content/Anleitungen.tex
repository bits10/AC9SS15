\chapter{Anleitungen}

\section{Das Atmel Studio}

\section{Einrichten eines neuen Microcotrollers}
Für unser Projekt sollen alle Notwendigen Scripte auf dem Microcotnroller
gespeichert werden. Der beim AVR-Net-IO mitgelieferte ATmega32 bietet hierfür
jedoch nicht ausreichend Speicher.
Wir haben uns deswegen für den aus der gleichen baureihe stammenden ATmega644P
entschieden der mit seinen 64KB Programmspeicher doppeltsoviel platz bietet wie
der kleinere ATmeag32.

Für einen neuen Chip ist es anfangs notwendig die Fuse-Bits richitg zu Setzen,
damit der Chip ordnugnsgemäß Arbeitet.
Dies ist jedoch im AtmelStudio nicht ohneweiteres Möglich.
Wenn man

\section{Configuration}

\section{HTML Header Compiler}

\section{Hexfiles Überspielen}

\section{Die Website}

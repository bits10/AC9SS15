\chapter{Der Webserver}

Als Basis für unser Projekt haben wir die Firmware von Ulrich Radig verwendet.
Da diese Vorlage für unseren Anwendungsfall zu umfangreich ist,  haben wir uns 
für die Abgespeckte Variante von Günther Menke entschieden. Die Änderungen sind
zum einen das entfernte Kamera-Feature und um zusätzlichen Quellcode für einen
alternativen Netzwerkcontroller abgespeckt wurde.

\section{Einrichtung}

Die Einstellung des Webservers erfolgt über die \textrm{config.h} Datei. Hier können
IP-Adresse, Mac-Adresse oder die Ports eingestellt werden.

\section{Einbindung der Website}

Alle für den Betrieb der Website benötigten Dateien sind nicht über ein
Dateisystem vorhanden, sondern werden beim zugriff des Benutzers auf den
Webserver über die \textrm{webpage.h} datei geladen.
Das erstellen der .h erfolgt über das Beigelegte \textrm{HTML Header Compiler}
Werkzeug erstellt. Eine Beispiel zum erstellen der \textrm{webpage.h} Datei gibt
es im Tutorial Abschnitt. Das Werkzeug wird im Kapitel Werkzeuge detailierter
erklärt. Abschließend ist noch zu erwähnen, das die \textrm{webpage.h} nicht für manuelle
Bearbeitung gedacht ist. Dies geschieht ausschließlich über die Quell-Dateien
und dem anschließenden umwandeln mit dem \textrm{HTML Header Compiler}



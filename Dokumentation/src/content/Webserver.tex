\chapter{Der Webserver}

Als Basis für unser Projekt haben wir die Firmware von Ulrich Radig verwendet.
Da diese Vorlage für unseren Anwendungsfall zu umfangreich ist,  haben wir uns 
für die Abgespeckte Variante von Günther Menke entschieden. Die Änderungen sind
zum einen das entfernte Kamera-Feature und um zusätzlichen Quellcode für einen
alternativen Netzwerkcontroller abgespeckt wurde.

\section{Einrichtung}

Die Einstellung des Webservers erfolgt über die \textrm{config.h} Datei. In der
\textrm{config.h} Datei, können Zum einen die verschiedenen Pins der Ports als Ein oder
Ausgang definiert werden. Dabei gibt es ein paar Eigenheiten zu beachten:
\begin{itemize}
  \item OUTA Steht für den A Port, hier ist du beachten, das dieser Port die
  Analog zu Digital Wandler beherbergt, Mit aktiviertem Wandler ist es nicht
  möglich die Pins des Ports funktionierend als Ausgänge zu schalten, da die
  Spannung nicht gehalten wird.
  \item OUTB Ist nur mit Vorsicht zu genießen, da es sich hier um den Port der
  für die Neztwerkkomunikation genutzt wird handelt. Deswegen wird der Port B
  auch nicht Standardmäßig definiert.
  \item OUTC Dieser Port wird von Polin standardmäßig für die Ausgänge verwendet
  und ist von uns bereits so Modifiziert. Der gesamte Port wird auf dem
  AVR-NET-IO über den 25Pin D-Sub Stecker geleitet. Wenn der Fuse-Bit für JTag
  geschaltet ist werden 4 Pins des C Ports für das JTag Interface vewendet.
  \item OUTC Der C Port liegt auf dem AVR-Net-IO auf dem EXT Anschluss und ist
  für erweiterte Präferie geplant, so kann hier ein Cardreader oder ein
  Erweiterungsboard angeschlossen werden.
\end{itemize}
Weiter kann die gewünschte IP-Adresse eingestellt werden, unter der das Gerät
erreicht werden kann. Wichtig ist hier, das kein anderes Gerät die selbe Adresse
im Netzwerk verwendet. Auch kann die Router IP-Adresse und Netzmaske angegeben
werden. Ferner gibt es noch weitere Funtionen, die vom Board unterstützt werden.
So ist es möglich geräte über Magic Packets mit Wake on Lan zu wecken, einen
DNS-Server zu verwenden oder einen NTP Server einzurichten.
%Ist das Getestet?
Ein weiterer Wichtiger Einstellungspunkt ist die Aktuelle Mac adresse des
Netzwerk Controllers. Diese wird über die Variablen MYMAC1-6 Definiert.
Abschließend gibt es noch einige module die Aktiviert werden können:
\begin{itemize}
  \item USE\_ADC
  \item HTTP\_AUTH\_DEFAULT
  \item HTTP\_AUTH\_STRING
  \item USE\_MAIL
  \item GET\_WEATHER
  \item CMD\_TELNET
\end{itemize}
%Testen oder Rauswerfen? [Jan]


\section{Einbindung der Website}

Alle für den Betrieb der Website benötigten Dateien sind nicht über ein
Dateisystem vorhanden, sondern werden beim zugriff des Benutzers auf den
Webserver über die \textrm{webpage.h} datei geladen.
Das erstellen der .h erfolgt über das Beigelegte \textrm{HTML Header Compiler}
Werkzeug erstellt. Eine Beispiel zum erstellen der \textrm{webpage.h} Datei gibt
es im Tutorial Abschnitt. Das Werkzeug wird im Kapitel Werkzeuge detailierter
erklärt. Abschließend ist noch zu erwähnen, das die \textrm{webpage.h} nicht für manuelle
Bearbeitung gedacht ist. Dies geschieht ausschließlich über die Quell-Dateien
und dem anschließenden umwandeln mit dem \textrm{HTML Header Compiler}



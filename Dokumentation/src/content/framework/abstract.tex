\chapter*{Abstract\markboth{Abstract}{}}
\addcontentsline{toc}{chapter}{Abstract}
Die Aufgabe des Semesterprojektes bestand darin, auf einem Pollin-Net-IO-Board, auf welchem als Prozessor ein ATmega644p 
läuft, einen Webserver aufzusetzen. Als Vorlage für den Webserver gab es eine Version von Ulrich Radig, welche man nutzen, verbessern und ausbauen soll.\\
Mit Hilfe des Webservers soll es möglich sein, den Status der sich auf dem Board 
befindenden Pins anzeigen und diese manipulieren zu lassen. Die Anzeige, sowie die Manipulation soll über eine Webseite, 
auf welcher das Board grafisch dargestellt wird geschehen. Zudem soll man pro Pin auf der Webseite eine Beschreibung und Funktion 
hinterlegen, welche gespeichert bleibt und abrufbar ist. Die Webseite soll den aktuellen Platinenstand grafisch anzeigen, Änderungen am Board sollen so 
schnell es möglich ist auf der Webseite grafisch angezeigt werden.\\

\chapter{Aufgabenstellung}

Die Aufgabe des Semesterprojektes bestand darin, einen Webserver auf einem Microcontroller aufzusetzen.\\
Bei der vorgegebenen Hardware handelt es sich um ein Pollin-Net-IO-Board, auf welchem als Prozessor ein ATmega644p 
läuft. Als Vorlage für den Webserver gab es eine Version von Ulrich Radig, welche man nutzen, verbessern und ausbauen soll.\\
Der Nutzer greift auf das Board über eine Webseite zu, auf welcher das Board grafisch angezeigt wird. Anhand dieser Grafik 
bekommt der Nutzer einen Überblick und kann den Status der Pins ablesen, sowie diese via Mausklick manipulieren.\\
Zudem ist es dem Nutzer möglich, pro Pin auf der Webseite eine Beschreibung und Funktion zu 
hinterlegen, welche gespeichert bleibt und somit bei einem Neustart wieder abrufbar ist. Somit kann sich der Nutzer kleine Skriptfunktionen und Pins, 
welcher er häufig nutzt als Favoriten setzen und mithilfe der eigenen Beschreibung schneller erfassen, welcher Pin welche Funktion hat.\\
Die Webseite soll den aktuellen Platinenstand grafisch anzeigen, Änderungen am Board sollen so 
schnell es möglich ist auf der Webseite grafisch angezeigt werden, genauso sollen Änderungen via Mausklickt sofort als Befehl 
an das Board übertragen und ausgeführt werden.\\
Mit dem ATmega664p ist der gesamte Speicherverbrauch (Also Server und Webseite) auf 64kB begrenzt, was die Herausforderung des  
Projektes ausmacht. Die Vorlage des Webservers von Ulrig Radig, an welche sich das Team halten und diese weiterentwickeln soll 
musste daher zunächst ausgemistet werden. Auch bei der Entwicklung der Webseite ist der Speicherverbrauch eine ständige Herausforderung.\\
Da das Projekt nach Erreichen der Aufgaben noch nicht fertig ist, sondern weiterentwickelt wird, muss der gesamte Code übersichtlich 
dokumentiert und gut nachvollziehbar sein.

\chapter{Rückblick}

\section{Soll/Ist-Vergleich}

\section{Verworfene Varianten}

\section{Eigenbewertung}


\subsection*{Christian Würthner}
Das Projekt lieferte einen guten Einblick in die Welt der Microcontroller und
der Webentwicklung. Wir konnten viele neue Techniken lernen und praktisch
anwenden. Eine so komplexe Webanwendung mit möglichst wenig Speicherverbrauch zu
entwickeln war eine große Herausforderung, es mussten immer Kompromisse zwischen
Komfort, Funktionalität und Speicherverbrauch getroffen werden. Auch im Bereich
Projektplanung haben wir wertvolle Erfahrungen gesammelt, die wir in das Projekt
im 6. Semester mitnehmen werden.

\subsection*{Jan-Henrik Preuß}
Trotz anfänglicher Schwierigkeiten hat mir das Projekt sehr viel Spaß gemacht.
Vor allem das für Informatiker eher ferne Aufgabengebiete der Elektrotechnik hat
mir einiges an Herausforderungen gegeben und mir gezeigt wie man mit wenig
Ressourcen ein doch umfangreiches Projekt zustande bekommt. Die Arbeit im Team 
hat mir gut gefallen und ich habe einiges an Praktischen Erfahrungen zur Planung
und Umsetzung von einem Semesterprojekt bekommen.

\subsection*{Ann-Sophie Dietrich}

\subsection*{Marcel Schlipf}
Das Projekt verlief meiner Meinung sehr gut. Die wöchentlichen Meetings gab uns
vor, in welche Richtung wir arbeiten mussten. In Team selbst sind wir sehr gut
miteinander klargekommen. Größere Probleme im Projekt sind wir auch mal zu zweit
oder dritt angegangen. \newline
Meine Kompetenzen haben sich in vor allem in Richtung Mikrokontroller erweitert.
Ich hatte noch nie etwas mit Mikrokontrollern am Hut, durch das Projekt hab ich
ein Einblick in diese Mikrowelt bekommen und kann auch schon mehr damit
anfangen.\newline
Da ich schon etwas erfahrung mit Webseiten enticklung habe sind nur noch
kleinigkeiten die ich dazugelernt habe.\newline
Das Semesterprojekt hat mir Spaß gemacht, man hat einfach einen Einblick
bekommen, wie Projekte im größeren Stile verlaufen könnten. Genauso kann man das
gelernte aus Projektmanagement anwenden.

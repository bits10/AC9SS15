
\chapter{Rückblick}

\section{Soll/Ist-Vergleich}
Das Projekt lief insgesamt gut nach dem zu Beginn angefertigten Plan(Siehe im Kapitel "`Projektplanung"'), jedoch gab es 
einige kleine Änderungen, welche aber keine großen Auswirkungen hatten. \\
Im folgenden werden die drei aufgetrenenen Probleme/Unstimmigkeiten kurz beschrieben und unsere 
Art, diese zu lösen.
\begin{center} % Tabelle mittig auf der Seite

\begin{longtable}{|>{\raggedright \arraybackslash}p{4.0cm}|
>{\raggedright \arraybackslash}p{4.0cm}|>{\raggedright \arraybackslash}p{4.0cm}|}

\hline
Aufgetretene Probleme & Beschreibung & Änderungen \\ \hline
Anforderung Falsch & Vorgegeben war die Arbeit mit Qt.\linebreak
Qt ist eine C++ IDE und nicht 
für die Arbeit an einer Webseite geeignet & Webseitenentwicklung mit JavaScript und HTML\\ \hline
%\linebreak \linebreak Man kann so auch Leerzeilen in den Text einfügen. \\
Zeitplan nicht eingehalten & Durch die Feldversuche mit dem Board rutschte das Team in die Implementierungsphase. 
\linebreak Nachdem Radig-asis lief, versuchten wir die Pins zu manipulieren und stellten fest, dass diese Versuche 
nicht mehr als "`Feldversuche"' zu deklarieren sind. & Meilenstein "Ende der Planungsphase" wurde 2 Wochen vorgelegt und 
als "`erreicht"' markiert.\linebreak Die Implementierungsphase wurde dadurch 2 Wochen länger. \\ \hline
Fehlende Ressource: \linebreak Debugger & Zu Beginn hatten wir einen reinen Programmer, dh jegliche Arbeit am Board war reines 
Try and Error. \linebreak Deshalb haben wir einen Debugger beantragt, auf welchen wir jedoch warten mussten, was zu einem 
Zeitverlust führte. & Arbeit an Board bis Erhalten des Debuggers try and Error \linebreak Durch die vielen Arbeitspakete fokussierten 
wir uns in dieser Wartezeit mehr auf die Webseite und hatten somit kaum Zeitverlust. \\ \hline
\caption{Soll/Ist Vergleich}
\end{longtable}
\end{center}

\section{Verworfene Varianten}
Uns fiel zunächst als Erweiterung die Ansteuerung des Boards über ein Raspberry Pi als Master ein.\\
Dieser Vorschlag wurde von der Projektverwalung zunächst zurückgewiesen, da eine serverbasierte Lösung
angefordert war. \\ Die Erweiterung hierfür behielten wir uns im Hinterkopf, konnten sie aber aufgrund 
von Zeitmangel nichtmehr umsetzten, weshalb man eine genauere Beschreibung hierfür im Kapitel "`Ausblick"'
findet.
\section{Eigenbewertung}


\subsection*{Christian Würthner}
Das Projekt lieferte einen guten Einblick in die Welt der Mikrocontroller und
der Webentwicklung. Wir konnten viele neue Techniken lernen und praktisch
anwenden. Eine so komplexe Webanwendung mit möglichst wenig Speicherverbrauch zu
entwickeln war eine große Herausforderung, es mussten immer Kompromisse zwischen
Komfort, Funktionalität und Speicherverbrauch getroffen werden. Auch im Bereich
Projektplanung haben wir wertvolle Erfahrungen gesammelt, die wir in das Projekt
im 6. Semester mitnehmen werden.

\subsection*{Jan-Henrik Preuß}
Trotz anfänglicher Schwierigkeiten hat mir das Projekt sehr viel Spaß gemacht.
Vor allem das für Informatiker eher ferne Aufgabengebiete der Elektrotechnik hat
mir einiges an Herausforderungen gegeben und mir gezeigt wie man mit wenig
Ressourcen ein doch umfangreiches Projekt zustande bekommt. Die Arbeit im Team 
hat mir gut gefallen und ich habe einiges an Praktischen Erfahrungen zur Planung
und Umsetzung von einem Semesterprojekt bekommen.

\subsection*{Ann-Sophie Dietrich}
Ich empfand das Projekt als lehrreich und hilfreich für das spätere Berufsleben.\\
Die regelmäßigen Meetings sowohl untereinander, als auch 
wöchentlichen Meetings mit Herrn Spale, gaben einen guten Einblick in das Berufsleben.\\
Innerhalb des Projektes habe ich ebenfalls viel gelernt, wie wichtig der Faktor Planung 
bei einem größeren Projekt ist, wie man Probleme bespricht und gemeinsam nach Lösungen sucht.
Auch fachlich hat mich dieses Projekt weitergebracht. Ich hatte zuvor keine Erfahrung 
mit JavaScript und bekam durch die Arbeit an der Webseite einen guten Einblick.\\
Die Erfahrungen, welche ich innerhalb des Projektes bekommen habe, werde ich bei 
späteren Projekten einsetzen.

\subsection*{Marcel Schlipf}
Durch das Projekt habe ich sehr viel neue Erkenntnis gewonnen. Angefangen in der
Welt der Mikrokontroller bis hin zu Planung. So deckten die wöchentlichen
Meetings neue Aufgaben und Probleme auf, die wiederum in Team erledigt, bzw.
behoben worden sind. In Team selbst sind wieder gut miteinander ausgekommen.\\
Ich konnte mich in der Welt der Mikrokontroller noch nicht zurechtfinden, aber
mein Wissen durch dieses Projekt erweitern und kann nun etwas mit diesen Geräten
anfangen. Die kleinere Erfahrung in der Webseitenentwicklung konnte ich in das
Projekt einbringen und ein wenig weiterentwicklen.\\
Die Erkenntnis aus diesem Projekt und aus der Vorlesung Projektmanagement werde
ich auch im 6.Semester im Projekt anwenden.

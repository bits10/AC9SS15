%Korrekturgelesen von : Ann-Sophie Dietrich
\chapter{Programmieren und Debuggen}

Um einen Mikrocontroller zu programmieren oder zu debuggen, gibt es zwei
Interfaces, diese werden hier einmal genauer beleuchtet.

\section{ISP}

Ein \acf{ISP} fähiger Mikrocontroller kann direkt in der Schaltung programmiert
werden, ohne entfernt zu werden. Programmieren kann man entweder mit dem
Programmer AVRISPmkII über die \acs{SPI} Schnittstelle oder mit dem  
AVRJTAGICEmkII. Der AVRJTAGICEmkII unterstützt zum Programmieren sowohl die
die \acs{SPI} als auch die \acs{JTAG} Schnittstelle.

\section{SPI}

\acf{SPI} ist die Schnittstelle, um den Mikrocontroller zu programmieren, über
diese Schnittstelle kann nicht debuggt werden. Neben einem Programmer, wie dem
AVRISPmkII, hängt an diesem Bus auch der ENC28J60 Netzwerk Controller.

\section{JTAG}

\acf{JTAG} ist eine Schnittstelle die neben der Programmierung von
Mikrocontrollern auch das Debuggen ermöglicht.
Dafür wird allerdings der relativ teure AVRJTAGICEmkII von Atmel benötigt.
In Verbindung mit dem Atmel Studio kann dann komfortabel getestet
werden. Wie beim Entwickeln von Programmen mit einer anderen Programmiersprache
können Breakpoints definiert werden und variabel ausgelesen werden. Für eine
umfangreiche Entwicklung wie unseren Webserver, bei dem viele dynamische
Ereignisse auftreten können, ist ein solches Programmierwerkzeug sehr hilfreich.

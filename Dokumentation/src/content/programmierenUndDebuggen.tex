\chapter{Programmieren und Debuggen}

Um einen Mikrocontroller zu Programmieren oder zu Debuggen gibt es zwei
Interfaces, diese Werden hier einmal genauer Beleuchtet.

\section{ISP}

Ein \ac{ISP} fähiger Mikrocontroller kann direkt in der Schaltung Programmiert
werden, ohne entfernt zu werden. Programmieren kann man entweder mit dem
Programmer AVRISPmkII über die \acs{SPI} Schnittstelle oder mit dem  
AVRJTAGICEmkII. Der AVRJTAGICEmkII unterstützt zum Programmieren sowhol die
die \acs{SPI} als auch die \acs{JTAG} Schnittstelle.


\section{SPI}



\section{JTAG}


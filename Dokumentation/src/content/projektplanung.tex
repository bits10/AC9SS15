\chapter{Projektplanung}

\section{Zeitlicher Ablauf}
Zu Beginn des Projektes mussten wir feststellen, dass einige Teammitglieder noch sehr 
unerfahren in der Welt der Microcontroller waren. Somit war es zunächst notwenig, sich mit 
den Grundlagen zu beschäftigen und sich in die Problematik einzulesen. Nach der ersten Gruppenbesprechung 
wurden Posten verteilt und ein grober Zeitplan erstellt. Schnell stellte sich 
heraus, dass wir ohne eine erste Besprechung und ohne die Platine nicht wissen, ob unsere 
Ideen und Vorschläge überhaupt umsetzbar sind, geschweige denn den Anforderungen entsprechen. \\
Nach der ersten Besprechung, in welcher wir die Platine überreicht bekamen, begannen die ersten
 Einarbeitungen mit dem Controller. Standardmäßig war eine Software beigelegt, mit welcher sich 
 bereits die Ein- und Ausgänge steuern ließen. \\
 Eine weitere Problematik lag darin, dass wir zwar einen \textbf{In-System-Programmer} (ISP) zum 
Anschluss der Platine an den PC hatten, doch war bei diesem Entwicklungswerkzeug
die falsche Pinbelegung vorhanden. Nach einiger Recherche fanden wir jedoch einige Anleitungen 
im Internet, welche hierbei für Klärung sorgten.\\
Die Standard-Ausführng des Controllers reichte jedoch nicht für ausreichendes testen, weshalb wir 
noch weiteres Zubehör anschaffen wollen.\\
Beim AVR-NET-IO sind die Digitalen ein und Ausgänge nur über den 25-Pin seriellen Eingang zu erreichen.
Deswegen wurde ein Bausatz angefordert, den wir auch umgehend von Herrn Schellhammer erhalten haben.
Mit diesem Bausatz können die digitalen Ausgänge direkt mit den Klemmen belegt werden.

Nachdem für den ISP Programmierer der Richtige Adapter gelötet wurde, konnten erste Tests mit dem Board gefahren werden
Zuerst wurde Testweise die Ethersex Firmware auf den Microcontroller aufgespielt und in betrieb genommen.
Für das Radig Projekt gab es allerdings noch ein paar Probleme, bevor die Software in Betrieb genommen werden konnte.


%Bevor wir mit dem eigentlichen Projekt beginnen konnten, 
%mussten wir zuerst einmal Grundlagenforschung betreiben.
%Als Problem tat sich heraus, das einige Team Mitglieder
%noch sehr unerfahren in der Welt der Microcontroller war.

%Nachdem wir in der ersten Besprechung die eigentliche Platine bekommen hatten,
%konnte die erste Einarbeitung in den Controller beginnen.
%Die Anfangsschwierigkeit lag darin, mit den angaben klar zu kommen 
%und welche ein und Ausgänge wie benutzt werden konnten, mit der 
%standardmäßig beigelegten Software.


%Als erstes Problem trat auf, das wir nicht die richtigen für die
%Entwicklung benötigten Geräte besaßen.
%So hatten wir zwar einen \textbf{In-System-Programmer} (ISP) zum 
%Anschluss der Platine an den PC doch war bei diesem Entwicklungswerkzeug
%die falsche Pinbelegung vorhanden.

%Doch hier konnten einige Anleitungen im Internet Abhilfe schaffen
%
%Des weiteren muss neben dem Controller noch weiteres Zubehör angeschafft werden,
%damit das Projekt ausreichend getestet werden kann.
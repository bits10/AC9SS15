\documentclass[ 
	12pt,
	a4paper,
	bibtotoc,
	cleardoubleempty, 
	idxtotoc,
	ngerman, 
	openright
	final, 
	listof=nochaptergap,
	]{scrbook}

\usepackage[T1]{fontenc}
\usepackage[utf8]{inputenc}
  
% ##################################################
% Unterstuetzung fuer die deutsche Sprache
% ##################################################
%\usepackage{ngerman}
\usepackage[ngerman]{babel}

% ##################################################
% Dokumentvariablen
% ##################################################

%Wird benötigt, resultiert sonst in einer Undefined control sequence
\newcommand{\docNachname}{ }
\newcommand{\docVorname}{ }

% Persoenliche Daten
\newcommand{\docA}{Jan-Henrik Preuß}
\newcommand{\docB}{Ann-Sophie Dietrich}
\newcommand{\docC}{Marcel Schlipf}
\newcommand{\docD}{Christian Würthner}

% Dokumentdaten
\newcommand{\docTitle}{Webserver für ein embedded Board mit AVR-Prozessor}
%\newcommand{\docUntertitle}{} % Kein Untertitel
\newcommand{\docUntertitle}{Dokumentation}
% Arten der Arbeit: Bachelorthesis, Masterthesis, Seminararbeit, Diplomarbeit
\newcommand{\docArtDerArbeit}{Projektarbeit}
%Studiengaenge: Allgemeine Informatik Bachelor, Computer Networking Bachelor,
% Software-Produktmanagement Bachelor, Advanced Computer Scinece Master
\newcommand{\docStudiengang}{AIB/CNB}
\newcommand{\docAbgabedatum}{30.07.2014}
\newcommand{\docErsterReferent}{Dr. Jiri Spale}
%\newcommand{\docZweiterReferent}{-} % Wenn es nur einen Betreuer gibt
%\newcommand{\docZweiterReferent}{ZWEITER REFERENT}

% ##################################################
% Allgemeine Pakete
% ##################################################

% Abbildungen einbinden
\usepackage{graphicx}

% Zusaetsliche Sonderzeichen
% \usepackage{dingbat}

% Farben
\usepackage{color}
\usepackage[usenames,dvipsnames,svgnames,table]{xcolor}

% Maskierung von URLs und Dateipfaden
\usepackage[hyphens]{url}

% Deutsche Anfuehrungszeichen
\usepackage[babel, german=quotes]{csquotes}

% Pakte zur Index-Erstellung (Schlagwortverzeichnis)
\usepackage{index}
\makeindex

% Paket zur gestaltung von Frames

\usepackage[framemethod=default]{mdframed}

\newmdenv[linecolor=red]{myframe}


% ##################################################
% Seitenformatierung
% ##################################################
\usepackage[
	portrait,
	bindingoffset=1.5cm,
	inner=2.5cm,
	outer=2.5cm,
	top=3cm,
	bottom=2cm,
	%includeheadfoot
	]{geometry}

% ##################################################
% Kopf- und Fusszeile
% ##################################################

\usepackage{fancyhdr}

\pagestyle{fancy}
\fancyhf{}
\fancyhead[EL,OR]{\sffamily\thepage}
\fancyhead[ER,OL]{\sffamily\leftmark}

\fancypagestyle{headings}{}

\fancypagestyle{plain}{}

\fancypagestyle{empty}{
  \fancyhf{}
  \renewcommand{\headrulewidth}{0pt}
}

%Kein "Kapitel # NAME" in der Kopfzeile
\renewcommand{\chaptermark}[1]{
	\markboth{#1}{}
   	\markboth{\thechapter.\ #1}{}
}

% ##################################################
% Schriften
% ##################################################

% Stdandardschrift festlegen
\renewcommand{\familydefault}{\sfdefault}

% Standard Zeilenabstand: 1,5 zeilig
\usepackage{setspace}
\onehalfspacing 

% Schriftgroessen festlegen
\addtokomafont{chapter}{\sffamily\large\bfseries} 
\addtokomafont{section}{\sffamily\normalsize\bfseries} 
\addtokomafont{subsection}{\sffamily\normalsize\mdseries} 
\addtokomafont{subsubsection}{\sffamily\normalsize\mdseries}
\addtokomafont{caption}{\sffamily\normalsize\mdseries} 

%Einrücken von Absätzen deaktivieren
\setlength{\parindent}{0pt}

% ##################################################
% Inhaltsverzeichnis / Allgemeine Verzeichniseinstellungen
% ##################################################

\usepackage{tocloft}

% Punkte auch bei Kapiteln
\renewcommand{\cftchapdotsep}{3}
\renewcommand{\cftdotsep}{3}

% Schriftart und -groesse im Inhaltsverzeichnis anpassen
\renewcommand{\cftchapfont}{\sffamily\normalsize}
\renewcommand{\cftsecfont}{\sffamily\normalsize}
\renewcommand{\cftsubsecfont}{\sffamily\normalsize}
\renewcommand{\cftchappagefont}{\sffamily\normalsize}
\renewcommand{\cftsecpagefont}{\sffamily\normalsize}
\renewcommand{\cftsubsecpagefont}{\sffamily\normalsize}

%Zeilenabstand in den Verzeichnissen einstellen
\setlength{\cftparskip}{.5\baselineskip}
\setlength{\cftbeforechapskip}{.1\baselineskip}

% ##################################################
% Abbildungsverzeichnis und Abbildungen
% ##################################################

\usepackage{caption}

\usepackage{wrapfig}

% Nummerierung von Abbildungen
\renewcommand{\thefigure}{\arabic{figure}}
\usepackage{chngcntr}
\counterwithout{figure}{chapter}

% Abbildungsverzeichnis anpassen
\renewcommand{\cftfigpresnum}{Abbildung }
\renewcommand{\cftfigaftersnum}{:}

% Breite des Nummerierungsbereiches [Abbildung 1:]
\newlength{\figureLength}
\settowidth{\figureLength}{\bfseries\cftfigpresnum\cftfigaftersnum}
\setlength{\cftfignumwidth}{\figureLength}
\setlength{\cftfigindent}{0cm}

% Schriftart anpassen
\renewcommand\cftfigfont{\sffamily}
\renewcommand\cftfigpagefont{\sffamily}

% ##################################################
% Tabellenverzeichnis und Tabellen
% ##################################################

% Nummerierung von Tabellen
\renewcommand{\thetable}{\arabic{table}}
\counterwithout{table}{chapter}

% Tabellenverzeichnis anpassen
\renewcommand{\cfttabpresnum}{Tabelle }
\renewcommand{\cfttabaftersnum}{:}

% Breite des Nummerierungsbereiches [Abbildung 1:]
\newlength{\tableLength}
\settowidth{\tableLength}{\bfseries\cfttabpresnum\cfttabaftersnum}
\setlength{\cfttabnumwidth}{\tableLength}
\setlength{\cfttabindent}{0cm}

%Schriftart anpassen
\renewcommand\cfttabfont{\sffamily}
\renewcommand\cfttabpagefont{\sffamily}

% Unterdrueckung von vertikalen Linien
\usepackage{booktabs}

%Fluss eigenschaften von Tabellen

\usepackage{float}

%Longtable

\usepackage{longtable}

% ##################################################
% Listings (Quellcode)
% ##################################################

\usepackage{listings}
\definecolor{codegreen}{rgb}{0,0.6,0}
\definecolor{codegray}{rgb}{0.5,0.5,0.5}
\definecolor{codepurple}{rgb}{0.58,0,0.82}
\definecolor{backcolour}{rgb}{0.95,0.95,0.92}
 
\lstdefinestyle{codestyle}{
    backgroundcolor=\color{backcolour},   
    commentstyle=\color{codegreen},
    keywordstyle=\color{magenta},
    numberstyle=\tiny\color{codegray},
    stringstyle=\color{codepurple},
    basicstyle=\footnotesize,
    breakatwhitespace=false,         
    breaklines=true,                 
    captionpos=b,                    
    keepspaces=true,                 
    numbers=left,                     
    numbersep=5pt,                  
    showspaces=false,                
    showstringspaces=false,
    showtabs=false,                  
    tabsize=2
}

\lstset{style=codestyle}

 \lstdefinelanguage{JavaScript}{
  keywords={break, case, catch, continue, debugger, default, delete, do, else, finally, for, function, if, in, instanceof, new, return, switch, this, throw, try, typeof, var, void, while, with},
  morecomment=[l]{//},
  morecomment=[s]{/*}{*/},
  morestring=[b]',
  morestring=[b]",
  sensitive=true
}	
% ##################################################
% Theoreme
% ##################################################
  	
% Umgebung fuer Beispiele
\newtheorem{beispiel}{Beispiel}

% Umgebung fuer These
\newtheorem{these}{These}

% Umgebung fuer Definitionen
\newtheorem{definition}{Definition}
  	
% ##################################################
% Literaturverzeichnis
% ##################################################

\usepackage{bibgerm}

% ##################################################
% Abkuerzungsverzeichnis
% ##################################################

\usepackage[printonlyused]{acronym}

% ##################################################
% PDF / Dokumenteninternelinks
% ##################################################

\usepackage[
	colorlinks=false,
   	linkcolor=black,
   	citecolor=black,
  	filecolor=black,
	urlcolor=black,
    bookmarks=true,
    bookmarksopen=true,
    bookmarksopenlevel=3,
    bookmarksnumbered,
    plainpages=false,
    pdfpagelabels=true,
    hyperfootnotes,
    pdftitle ={\docTitle},
    pdfauthor={\docVorname~\docNachname},
    pdfcreator={\docVorname~\docNachname}]{hyperref}
 
\begin{document}

\setcounter{secnumdepth}{3}
 
% Titelblatt
\include{content/framework/title}
\cleardoubleemptypage

\frontmatter



% Abstract
\chapter*{Abstract\markboth{Abstract}{}}
\addcontentsline{toc}{chapter}{Abstract}
Die Aufgabe des Semesterprojektes bestand darin, auf einem Pollin-Net-IO-Board, auf welchem als Prozessor ein ATmega644p 
läuft, einen Webserver aufzusetzen. Als Vorlage für den Webserver gab es eine Version von Ulrich Radig, welche man nutzen, verbessern und ausbauen soll.\\
Mit Hilfe des Webservers soll es möglich sein, den Status der sich auf dem Board 
befindenden Pins anzeigen und diese manipulieren zu lassen. Die Anzeige, sowie die Manipulation soll über eine Webseite, 
auf welcher das Board grafisch dargestellt wird geschehen. Zudem soll man pro Pin auf der Webseite eine Beschreibung und Funktion 
hinterlegen, welche gespeichert bleibt und abrufbar ist. Die Webseite soll den aktuellen Platinenstand grafisch anzeigen, Änderungen am Board sollen so 
schnell es möglich ist auf der Webseite grafisch angezeigt werden.\\

\cleardoubleemptypage

% Inhaltsverzeichnis
\tableofcontents
\addcontentsline{toc}{chapter}{Inhaltsverzeichnis}
\cleardoubleemptypage

% Abbildungsverzeichnis einbinden und ins Inhaltsverzeichnis
% WORKAROUND: tocloft und KOMA funktionieren zusammen nicht
% korrekt\phantomsection
\addcontentsline{toc}{chapter}{\listfigurename} 
\listoffigures
\cleardoubleemptypage

% Tabellenverzeichnis einbinden und ins Inhaltsverzeichnis
% WORKAROUND: tocloft und KOMA funktionieren zusammen nicht
% korrekt\phantomsection
\phantomsection
\addcontentsline{toc}{chapter}{\listtablename}
\listoftables
\cleardoubleemptypage

% Abkürzungsverzeichnis
\chapter*{Abkürzungsverzeichnis\markboth{Abkürzungsverzeichnis}{}}
\addcontentsline{toc}{chapter}{Abkürzungsverzeichnis}

\begin{acronym}
\acro{HFU}{Hochschule Furtwangen University}
\acro{ISP}{In System Programming}
\acro{SPI}{Serial Peripheral Interface}
\acro{JTAG}{Joint Test Action Group}
\end{acronym}


\mainmatter

\chapter{Einleitung}

Durch die weiter fortschreitende Vernetzung unserer alltäglichen Elektronik
wird das verlangen nach netzwerkfähigen Mikrocontrollern
immer wichtiger. Wenn der moderne Kühlschrank erkennen soll, wie gut er gerade
gefüllt ist oder wenn die Heizung melden soll wie warm das Wasser ist, muss ein
entsprechend ausgestatteter Controller diese Informationen an den Nutzer
melden können. Hier greift unser Projekt da mit dem Mikrocontroller und seinen
verschiedenen ein und Ausgängen unterschiedlichste Anwendungen ermöglicht
werden. 

\chapter{Aufgabenstellung}

Die Aufgabe des Semesterprojektes bestand darin, einen Webserver auf einem Microcontroller aufzusetzen.\\
Bei der vorgegebenen Hardware handelt es sich um ein Pollin-Net-IO-Board, auf welchem als Prozessor ein ATmega644p 
läuft. Als Vorlage für den Webserver gab es eine Version von Ulrich Radig, welche man nutzen, verbessern und ausbauen soll.\\
Der Nutzer greift auf das Board über eine Webseite zu, auf welcher das Board grafisch angezeigt wird. Anhand dieser Grafik 
bekommt der Nutzer einen Überblick und kann den Status der Pins ablesen, sowie diese via Mausklick manipulieren.\\
Zudem ist es dem Nutzer möglich, pro Pin auf der Webseite eine Beschreibung und Funktion zu 
hinterlegen, welche gespeichert bleibt und somit bei einem Neustart wieder abrufbar ist. Somit kann sich der Nutzer kleine Skriptfunktionen und Pins, 
welcher er häufig nutzt als Favoriten setzen und mithilfe der eigenen Beschreibung schneller erfassen, welcher Pin welche Funktion hat.\\
Die Webseite soll den aktuellen Platinenstand grafisch anzeigen, Änderungen am Board sollen so 
schnell es möglich ist auf der Webseite grafisch angezeigt werden, genauso sollen Änderungen via Mausklickt sofort als Befehl 
an das Board übertragen und ausgeführt werden.\\
Mit dem ATmega664p ist der gesamte Speicherverbrauch (Also Server und Webseite) auf 64kB begrenzt, was die Herausforderung des  
Projektes ausmacht. Die Vorlage des Webservers von Ulrig Radig, an welche sich das Team halten und diese weiterentwickeln soll 
musste daher zunächst ausgemistet werden. Auch bei der Entwicklung der Webseite ist der Speicherverbrauch eine ständige Herausforderung.\\
Da das Projekt nach Erreichen der Aufgaben noch nicht fertig ist, sondern weiterentwickelt wird, muss der gesamte Code übersichtlich 
dokumentiert und gut nachvollziehbar sein.

\chapter{Team}

\section{Teammitglieder}

\subsection*{Jan Henrik Preuß}

\subsection*{Christian Würthner}

\subsection*{Ann-Sophie Dietrich}

\subsection*{Marcel Schlipf}
\chapter{Projektplanung}

\section{Zeitlicher Ablauf}

%Issue #11 Beachten

\begin{quote}
	\textit{
		\enquote{
		zu viel Prosa
		für wen ist dieser Text bestimmt? - (Doku, was gemacht wurde; Leute die es nachbauen oder weiterentwicklen möchten)
		Das ist ein studentisches Projekt eine Einarbeitung benötigt, ist normal
		wer bestimmte Teile organisiert hat, ist nicht wichtig;
		Text prägnanter formulieren
		Fehlt: Meilensteine, Grafiken
		}
	}
	\cite{Spale.2014}
\end{quote}



\chapter{Hardware}
%http://samurai1967.dyndns.org/avr-net-io.html
%http://www.fhemwiki.de/wiki/AVR-NET-IO
%http://www.mikrocontroller.net/articles/AVR_Net-IO_Bausatz_von_Pollin
\section{AVR Net-IO-Board}
\begin{figure}[h]
\centering
\includegraphics[width=10cm]{content/pictures/avr-net-io.jpg}
\caption{AVR-NET-IO - Pollin GmbH}
%http://www.pollin.de/shop/dt/MTQ5OTgxOTk-/Bausaetze_Module/Bausaetze/Bausatz_AVR_NET_IO.html
\label{fig:B3}
\end{figure}

\subsection{Technische Daten}
\begin{itemize}
  \item Betriebsspannung 9V
  \item Stromaufnahme ca. 190 mA
  \item 8 Digitale Ausgänge, 4 Digitale Eingänge
  \item 4 Analoge Eingänge
  \item ATmega32 Mikrocontroller
  \item integrierte ISP-Schnittstelle
\end{itemize}

\section{Mikrocontroller}
\subsection{ATmega32}

\subsection{ATmega644P}

\subsection{ATmega1284P}

\section{Fuse Bits}

%TODO Wahl der Fuse Bits

\chapter{Programmieren und Debuggen}

Um einen Mikrocontroller zu Programmieren oder zu Debuggen gibt es zwei
Interfaces, diese Werden hier einmal genauer Beleuchtet.

\section{ISP}

Ein \ac{ISP} fähiger Mikrocontroller kann direkt in der Schaltung Programmiert
werden, ohne entfernt zu werden. Programmieren kann man entweder mit dem
Programmer AVRISPmkII über die \acs{SPI} Schnittstelle oder mit dem  
AVRJTAGICEmkII. Der AVRJTAGICEmkII unterstützt zum Programmieren sowhol die
die \acs{SPI} als auch die \acs{JTAG} Schnittstelle.


\section{SPI}



\section{JTAG}


\chapter{Recherche}

\section{Andere Lösungsmöglichkeiten}

Neben dem vorgegebenen Projekt von Ulrich Radig gibt es noch einige andere
Möglichkeiten einen Webserver oder Scripte auf das AVR-Net-IO zu bekommen.

\subsection{Ethersex}

Auf der Projekt-Website (\url{http://ethersex.de}) bewirbt sich Ethersex als
\begin{quote} \textit{
		\enquote{[\ldots] eine Firmware mit Netzwerkunterstützung für 8-bit AVR
		Mikrokontroller, die durch eine Community entwickelt wird.  }
	}
	\cite{Ethersex}
\end{quote}

Sie unterstützt sowohl die von uns verwendeten ATMega Prozessoren, das
AVR-Net-IO board und auch den verwendeten ENC28J60 Netzwerk Controller.
Durch einen ausführlichen Quick Start Guide
(\url{http://ethersex.de/index.php/Quick_Start_Guide/Preparation}) auf der
Projekt Seite ist es auch sehr einfach möglich selbst für Laien einen
lauffähiges Projekt hinzugekommen. Fast die gesamte Konfiguration kann
über einen Menü vorgenommen werden. So muss nicht wie beim Radig Projekt zum
ändern der Mac Adresse in eine Headerdatei geschrieben werden. 

\begin{figure}[H]
	\centering
		\includegraphics[width=13cm]{content/pictures/Recherche/Ethersex/Ehtersex1.png}
	\caption{Ethersex menuconfig}
	\label{Ethersex1}
\end{figure} 

Im Bild (Abb. \ref{Ethersex1}) sieht man den Startbildschirm des
Menüs, das über den Befehl \textrm{make menuconfig} erreicht werden kann.
Hier können verschiedenste Einstellungen getroffen werden. Zum Beispiel, den
verwendeten Microkontroller, welche Mac Adresse der Netzwerk Controller
verwendet oder welche IP Adresse gewünscht ist.
Nachdem die Konfiguration abgeschlossen ist, kann das Hexfile mit dem
\textrm{make} Befehl erstellt werden. In der Abbildung \ref{Ethersex2} sieht man
das am ende des \textrm{Make} Prozesses die aktuelle Größe des erstellten Binary Datei
angezeigt ist.

\begin{figure}[H]
	\centering
		\includegraphics[width=13cm]{content/pictures/Recherche/Ethersex/Ethersex2.png}
	\caption{Ethersex make project}
	\label{Ethersex2}
\end{figure} 

Das Einbinden der Website beim Ethersex Projekt wird in der Anleitung
folgendermaßen beschrieben:

\begin{quote}
	\textit{
		\enquote{Falls die Option Supply Inline Files aktiviert ist, werden alle
		Dateien, die unter vfs/embed/ abgelegt sind, automatisch beim Erstellen des
		Images mit gzip gepackt und an das Ende der Firmware angehängt. Die
		Dateinamen bleiben dabei unverändert [\ldots]} }
	\cite[\url{http://www.ethersex.de/index.php/HTTPD_(Deutsch)}]{Ethersex}
\end{quote}

\subsection{Elektronik 2000}

Einen anderen Ansatz verfolgt das Projekt Elektronik 2000.
Hier wird nicht nur der Webserver geboten sondern eine erweiterte GUI um das
Board zu programmieren. Dafür wird mit einem grafischem Designer eine Logik
entworfen und über einen ISP Programmer auf das Board gebracht.

\begin{quote}
	\textit{
		\enquote{Das E2000-NET-IO basiert auf dem AVR-NET-IO von Pollin. Durch die
		E2000-Firmware wird aus dem AVR-NET-IO von Pollin ein autak laufendes
		Logikmodul. Mit diesem Modul können über Netzwerk Schaltvorgänge ausgeführt
		werden. Außerdem sind Zeitgesteuerte Schaltvorgänge möglich.} }
	\cite{elektronik2000}
\end{quote}

Durch die Netzwerkanbindung des AVR-Net-IO kann dann die Programmierte Logik von
außen überwacht und gesteuert werden. Dafür gibt es von den Entwicklern eine
bereitgestellte Android Applikation. Zusätzlich zu einem Projekt das mit dem
AVR-Net-IO arbeitet gibt es mittlerweile eine weitere Version die auch mit dem
Raspberry Pi zusammenarbeitet und die GPIO Pins des Pis nach außen steuerbar
macht.



\chapter{Ausgewählte Lösung}
%TODO überarbeiten issue #18

Neben den Beiden anderen Projekten haben wir uns für das Projekt von Ullrich
Radig Entschieden. Die Vorteile des Projektes gegenüber Ethersex oder Elektronik
2000 liegen darin, das die beiden Projekte zu speziell und umfangreich für unsere
Anforderungen sind. Zum einen war es unsere Aufgabe eine möglichst umfangreiche
Website zu erstelle, die Struckur von Ethersex ist für eine auf die Website
fokusierte Programmierung schlicht zu umfangreich und schlechter anpassbar.

\section{Der Webserver}
%TODO überarbeiten issue #18

Als Basis für unser Projekt haben wir die Firmware von Ulrich Radig verwendet.
Zusätzlich von der Uhrsprungsversion von Ulrich Radig gibt es noch eine Etwas
vereinfachte Version von Günther Menke. Wir haben wir uns für die vereinfachte
Variante von Günther Menke entschieden. Die Unterschiede zwischen beiden
Versionen belaufen sich auf das entfernte Kamera-Feature und um
zusätzlichen Quellcode für einen alternativen Netzwerkcontroller.

\subsection{Änderungen}

Zu der bereits vereinfachten Version von Günther Menke mussten wir für unser
Projekt noch Funktionalität von der ursprünglichen Version entfernen. Das
Problem lag darin, das wir für die Dateien der Website möglichst viel freien
Speicherplatz benötigen, der von den entsprechenden Funktionen belegt wurde.
Schlussendlich wurde Folgende Funktionalität aus der Version von Günther Menke
entfernt:

\begin{itemize}
  \item \textbf{(WOL) Wake on Lan} Funktionalität um andere Geräte im Netzwerk
  durch bestimmte Datenpackete aufzuwecken.
  \item \textbf{Sendmail} Senden von E-Mails.
  \item \textbf{Weather} Ermitteln von Wetterdaten.
  \item \textbf{(NTP) Network Time Protocol} Empfangen von Internetzeit
  \item \textbf{(DNS) Domain Name System} Beantwortung von Anfragen zur
  Namensauflösung.
  \item \textbf{(USART) Universal Synchronous and Asynchronous Serial Receiver and
  Transmitter} eine Schnittstelle im Mikrocontroller zum Daten
  Austausch mit PC über die COM-Schnittstelle.
  \item \textbf{(Telnet) Telecommunication Network} zeichenorientierten
  Datenaustausch über eine TCP-Verbindung.
  \item \textbf{(CMD) Command Control} Verwaltung der Telnet Konsolen Befehle.
\end{itemize}

\subsection{Einbindung der Website}

Die Website, welche hauptsächlich aus verschiedenen .html und .js Dateien
besteht, ist mangels Dateisystem für unsere Firmware nicht verwendbar. Die
gesamten Dateien müssen in einer C-Headerdatei gebunden werden.
Das erstellen der Headerdatei erfolgt über das beim Projekt beigelegte
\ac{HHC} Werkzeug. Eine Beispiel zum erstellen der \textrm{webpage.h} 
Datei und eine Erklärung des \ac{HHC} gibt es im Kapitel Werkzeuge
\ref{chap:hintergrund.HHC}. Eine Anleitung zur Ausführung des \ac{HHC} gibt es im
Benutzerhandbuch \ref{chap:benutzerhandbuch.HHC}.
Abschließend ist noch zu erwähnen, das die \textrm{webpage.h} nicht für manuelle
Bearbeitung gedacht ist. Dies geschieht ausschließlich über die Quell-Dateien
und anschließendem umwandeln mit dem \ac{HHC}.

\section{Die Website}

Der Aufbau der Webseite ist in mehrer Dateien aufgeteilt. So haben wir mehrer
.js und .css Dateien. In der index.html wird alles nur zusammengetragen.
\newline


\chapter{Technischer Hintergrund}

\section{Vorgeschlagene Lösung}

\subsection{Kommunikation im Projekt Radig}
Im Projekt Radig ist keine echte Kommunikation zwischen dem Client und dem Server
vorhanden.

Die auf der Webseite dargestellten Werte werden vor dem senden der HTML-Seite
im HTML-Code eingefügt indem Platzhalter im Format "`\%PORTA0"' ersetzt und so statisch auf
der Webseite dargestellt werden. Das Manipulieren der Pins findet über ein HTML-Formular
statt. Alle manipulierbaren Pins sind als Input vom Typ Checkbox dargestellt. Diese lassen
sich frei manipulieren und erst beim Betätigen des "Senden"-Buttons werden die
Informationen per POST-Event an den Server gesendet und so die Seite neu aufgerufen. Der
Server filtert die POST Informationen aus dem HTTP-Header und manipuliert die Pins gemäß
den Anweisungen. Beim senden des angeforderten HTML-Dokumentes werden die neuen Werte in
den HTML-Code eingefügt, und so die neuen Werte auf der Webseite angezeigt.

Das große Problem bei dieser technisch einfachen Lösung ist, das geänderte Werte erst beim
nächsten neu laden der Webseite angezeigt werden. Ändert sich ein Pin während die Webseite
dargestellt wird bekommt der Nutzer dies nicht mit. Zudem wird bei jedem Manipulieren
eines Pins die gesamte Seite neu geladen und so Unmengen an unnötigen Daten übertragen.
Auch zum darstellen der aktuellen Werte muss die ganze Seite neu vom Server angefordert
werden.

\subsection{Erster Ansatz}
Die Kommunikation zwischen Server und Client sollte mit Hilfe einer REST-Schnittstelle
stattfinden, die im Hintergrund über Javascript angesprochen werden kann.

Eine REST-Schnittstelle besteht aus einer oder mehreren virtuellen URLs. Beim Aufruf einer
solchen URL liefert der Server kein Dokument das gespeichert ist, sondern erzeugt
dynamisch eine Antwort mit den benötigten Informationen und sendet diese als Antwort
zurück. Der Server kann beim Aufruf einer URL auch eine Aktion ausführen.

Vorteile der REST-Schnittstelle ist die simple Implementierung, sowohl auf dem Client mit
JavaScript als auch auf dem Server. Die Inhalte werden mit JSON formatiert, welches einen
technisches Standart darstellt und sich in JavaScript direkt in ein Objekt umwandeln
lässt. Auf dem Server ist es einfach mit einem Stringformat immer gleiche JSON Strukturen
zu erstellen und nur aktuelle Werte einzufügen. Die REST-Schnittstelle lässt sich leicht
um weitere, neue Funktionalitäten erweitern, indem neue virtuelle URLs erstellt werden die
vom Client ansprechbar sind.

Die Anforderungen an eine Lösung in diesem Projekt waren vor allem eine möglichst kompakte
Schnittstelle zu schaffen die wenig Bandbreite verbraucht um eine hohe
Übertragungsgeschwindigkeit zu ermöglichen trotz des schwachen Servers. Ein besonderes
Augenmerk war auf die Übertragung der Messwerte zu legen, da diese nicht wie andere
statische Informationen nur einmalig übertragen werden sondern kontinuierlich erneuert
werden müssen. Die Schnittstelle sollte gut skalierbar sein. Würde später ein Port für
eine andere Aufgabe zu verwendet werden muss dieser Port ohne Aufwand aus der
REST-Schnittstelle ausgeschlossen werden können, damit er von außen nicht manipulierbar
ist und so interne Abläufe auf der Platine nicht gestört werden.

Nach den Anforderungen muss die Schnittstelle folgende Aufgaben ermöglichen:
\begin{itemize}
  \item Abfragen der aktuellen Werte aller verwendbaren Pins
  \item Abfragen der Konfiguration eines Pins (Eingang oder Ausgang)
  \item Abfragen von Allgemeinen Informationen des Boards (IP, Standart-IP, Mac-Adresse,
  Serverversion)
  \item Manipulieren aller als Ausgänge geschaltener Pins
  \item Manipulieren der Konfiguration eines Pins (als Eingang oder Ausgang setzen)
  \item Manipulieren von Servereinstellungen (z.B. IP-Adresse);
\end{itemize}

%\begin{itemize}
%	\item Aufbau als REST-Schnittstelle
%	\begin{itemize}
%		\item Simpel implementierbar
%		\item Technischer Standart
%		\item Fertige Mechaniken in JavaScript
%		\item Leicht erweiterbar um neue Funktionalitäten
%	\end{itemize}
%	\item Anforderungen an eine Lösung:
%	\begin{itemize}
%		\item Möglichst kompakt, wenig Bandbreite soll verbraucht werden
%		\item Messwerte müssen besonders effizient übertragen werden
%		\item Skalierbar - Egal ob ein oder 20 Ports verwendet werden
%		\item Neben Abfragen der Messwerte auch manipulieren der IP, des DDR und der
%			  Ausgänge
%		\item JSON-Encodiert, direkt in JavaScript Objekt überführbar
%	\end{itemize}
%\end{itemize}

%-----------------------------------------------------------------------------------------
\subsection{Polling oder Pushing}
Die aktuellen Werte der Pins müssen bei jeder Änderung vom Server zum Client übertragen
werden, damit diese auf der Webseite immer korrekt dargestellt werden. Hierfür stehen zwei
verschiedene Konzepte zur Verfügung wie die Übertragung der Daten initialisiert werden.

\subsubsection{Polling}
Bei Polling werden vom Client kontinuierlich die Werte erneut angefordert, indem dieser
die entsprechende virtuell URL des Servers aufruft. Dies führt dazu, dass viele unnötige
Date übertragen werden, da sich eventuell nicht bei jedem erneuten anfordern der Werte
diese auch tatsächlich verändert haben und so die gleichen Datensätze oft mehrmals
angefordert werden.

Im vergleich zu der Radig-Lösung bietet Polling den Vorteil das die Werte kontinuierlich
nach geladen und so immer korrekt dargestellt werden während die Webseite dargestellt
wird. Auch das gesendete Datenvolumen wird dahingehend minimiert, das nur die Nutzdaten
übertragen werden und nicht der gesamte HTML-Code der Webseite. Polling ist technisch sehr
einfach zu realisieren, da die Abfrage der Daten einfach zyklisch wiederholt werden.

\subsubsection{Pushing}
Bei Pushing wird im Gegensatz zu Polling der Daten nicht vom Client initialisiert sondern
vom Server. Der Server weiß wann sich die Werte geändert haben und kann dem Client bei
jeder Änderung gezielt die neuen Daten übermitteln. Das Übertragen der Daten
könnte z.B. durch einen Interrupt ausgelöst werden.

Im direkten vergleich zu Polling bietet Pushing verschiedene Vorteile. So wird nicht nur
das Volumen der übertragenen Daten reduziert indem keine unnötigen Abfragen stattfinden,
sondern die neuen Werte gelangen auch genau dann zum Client wenn die Änderung tatsächlich
stattgefunden hat, was dazu führt das die Webseite schneller auf Änderungen reagiert.

Die technische Umsetzung von Pushing ist mit diversen Problemen verbunden. Die typische
Verbindungsaufbaurichtung ist bei Webanwendungen und Webseiten immer vom Client zum
Server. Anders als bei Polling müssen bei Pushing Daten vom Server zum Client gelangen.
Hierfür muss eine Verbindung vom Server zum Client aufgebaut werden. Dies ist technisch
aber nicht möglich, da der Browser bzw. JavaScript keine Möglichkeit haben einen Port des
Clientsystems zu öffnen und auf eingehende Verbindungen des Servers zu antworten.

Das Problem lässt sich durch die Benutzung von HTML5 Server-Sent Events umgehen. Hierbei
frägt der Client eine virtuelle URL des Servers ab, ähnlich einer REST-Schnittstelle. Der
Server überträgt jedoch nicht sofort Daten, sondern schreibt erst bei einem Event (z.B.
die Änderung eines Pins) in den geöffneten Stream und pusht so die Daten zum Client.
Dieser überwacht den Stream mit Hilfe von JavaScript un empfängt so die neuen Werte und
kann sie auf der Webseite anzeigen.

Dieses System ist auf dem Pollin Net-IO Board aber nur schwer umzusetzen da mehrere
Verbindungen verwaltet werden müssen. So ist immer mindestens eine Server-Sent Event
Verbindung offen, parallel könnte aber ein Client andere Daten vom Server anfordern. Für
das Verwalten mehrerer Verbindungen sind aber viele Resourcen nötig, da für jede
Verbindung auch Daten im RAM hinterlegt werden müssen. Außerdem st in vielen Situationen
ein simples Multitasking nötig, das so auf einem ATmega CPU nicht vorhanden ist. Das Radig
Projekt setzt aus diesen Gründen auf HTTP 1.0 bei dem für jede Anfrage eine Verbindung
geöffnet und nach erfolgreichem Übertragen der Daten wieder geschlossen wird. So ist auch
die Kommunikation mit mehreren Clients problemlos möglich.

\subsubsection{Entscheidung}
Pushing ist leider nur schwer umsetzbar auf dem Microcontroller. Dies ist
bedingt durch die knappen Resourcen wie Arbeitsspeicher, als auch durch das nur
schwer umsetzbare Multitasking, das für einen reibungslosen Ablauf nötig ist. Ohne Multitasking
ist nicht gewährliestet das alle Dateien übertragen werden können, da der
Server, sobald eine Server-Sent Event Verbindung aufgebaut wird, nicht mehr
verfügbar ist, bis diese wieder geschlossen wird. Eine Benutzung durch mehrere
Clients ist so nicht denkbar und auch die Nutzung von nur einem Client kann zu
Problemen führen wenn dieser Daten anfordert nachdem die Server-Sent Event
Verbindung aufgebaut worden ist.

Die Implementierung von Multitaskign ist zwar prinzipiell möglich, erfordert
aber einen tiefen Eingriff in das Vorlagenprojekt, da neben der eigentlichen
Nebenläufigkeit zusätzlich der gesamte Server threadsafe gestaltet werden
müsste.

Aus diesen Gründen entschlossen wir uns das technisch unsauberere Polling zu
verwenden, da die Implementierung von Pushing den Rahmen des Projektes
übersteigen würde.


%\begin{itemize}
%	\item Daten können entweder per Polling vom Client (Webseite abgefragt werden) oder
%		  vom Server bei einem Event (Änderung eines Eingangs) per Push geschickt werden
%	\item Pro/Contra Push
%	\begin{itemize}
%		\item Pro: Daten werden nur übertragen wenn es wirklich nötig ist, kein unnötiger
%			  Datenverkehr
%		\item Pro: Leistung bleibt auch mit mehreren Clients eher konstant
%			  (Genau Begründung ausarbeiten!)
%		\item Pro: Board wird entlastet, kein DOS, mehrere Clients können Webseite
%		      trotzdem problemlos aufbauen
%	\end{itemize}
%	\item Pro/Contra Polling
%	\begin{itemize}
%		\item Pro: Leichter zu implementieren - normale Dateiabfrage
%		\item Contra: Es werden viele unnötige Daten übertragen
%		\item Contra: Leistung nimmt mit steigender Anzahl von Clients ab
%		\item Pro: Im Prinzip nicht langsamer als Push: Limitierende Komponente ist die
%			  Übertragungszeit! (Hier Bild von Chrome Netzwerkvehrkehr)
%	\end{itemize}
%
%	\item Push bessere Lösung
%	\item Umsetzung aber nicht möglich
%	\item Server kann keine Verbindung zu Client aufbauen
%	\begin{itemize}
%		\item Javascript kann keinen Serverport öffnen
%		\item Client hat keine öffentliche IP
%	\end{itemize}
%	\item Technische Lösung: HTML 5 Server Sent Events
%	\begin{itemize}
%		\item Technischer Standart, Eingeführt mit HTML 5
%		\item Client ruft virtuelle URL auf, ähnlich REST
%		\item So wird ein Stream geöffnet
%		\item Bei einem Event schreibt der Server die Informationen in den Stream
%		\item Client benutzt Javascript API um die jeweiligen Informationen zu lesen
%	\end{itemize}
%	\item Mit ATmega CPU und Radig Projrkt als Vorlage nicht lösbar
%	\begin{itemize}
%		\item Radig kann nur eine Verbindung handeln. Aufbau - Übertragung - Abbau für
%			  jede Datei
%		\item HTTP 1.0
%		\item Aufruf einer HTML5 Server Sent Event URL würde dazu führen das der Server
%			  blockiert ist
%		\item Gesamte Struktur auf nur eine Verbindung ausgelegt
%		\item Bei Umschreiben (sehr tiefer Eingriff!) stößt die CPU schnell an Grenzen
%		\item Multitasking nötig, Stack für jede Verbindung führt zu RAM Problemen
%		\item Sprengt vermutlich zeitlichen Rahmen
%	\end{itemize}
%	\item Daraus resutltiert die Verwendung von Polling
%	\item Da Polling nicht langsamer ist, gibt es keine Nachteile bei Frequenzen von ca.
%		  200ms (auch bei zB. 3 Clients)
%\end{itemize}

%-----------------------------------------------------------------------------------------
\subsection{Aufbau der Server-Kommunikation}

Bei der Server-Kommunikation gibt es zwei grundsätzliche Kanäle:
\begin{itemize}
  \item Das Abfragen von Daten beim Server
  \item Das Manipulieren von Servereinstellungen (z.B. Pinwerte)
\end{itemize}
Das Manipulieren von Servereinstellungen war bereits im Projekt Radig möglich.
Hierfür interpretiert der Server die POST-Parameter jeder Anfrage und setzt ggf.
die Pins neu. Für die neuen Anforderungen wie das Manipulieren des DDR haben
wir und dazu entschlossen den bereits vorhandenen Code nur leicht zu
manipulieren und zu erweitern.

Für das Abfragen von Daten ist im Projekt Radig keine Lösung vorhanden, da hier
die Werte statisch in Form von Platzhaltern im HTML-Text eingebunden sind und
beim Abfragen der HTML-Datei durch die Werte ersetzt werden. Folglich muss die
gesamte Seite erneut geladen werden, um die dargestellten Werte zu
aktualisieren. Um dies zu vermeiden haben wir uns entschlossen die Daten
dynamisch abzufragen, um sie gesondert von der Webseite laden zu können.

\subsubsection{Mögliche Techniken für Datenabfrage}

Für die dynamische Datenabfrage gibt es veschiedene Standards. Hierzu gehöhren
verscheidene \ac{XML}-Basierte Protokolle wie RSS, als auch das so genannte
\ac{REST}.

\ac{XML}-Basierte Protokolle wie \ac{RSS} oder \ac{SOAP} weisen typischerweise
einen großen Protokoll-Overhead auf und sind deswegen nur bedingt für den Einsatz auf einem
Microcontroller geeignet, da das System zu viel unnötige Daten übertragen
müsste. Außerdem muss das dynamische \ac{HTTP}-Abfragen der Daten per JavaScript
erfolgen, welches die in \ac{XML} präsentierten Daten erst wieder parsen müsste, was
zusätzlichen Code auf dem Client bedeuten würden.

\ac{REST} hingegen präsentieren die Daten in \ac{JSON}, welches von
JavaScript direkt als Objekt interpretiert werden kann. Dies erspart das
aufwendige parsen der Daten. Außerdem ist der Protokoll-Overhead bei \ac{JSON}
tendenziell kleiner als bei \ac{XML}-Basierten Lösungen, da weniger lange Tagnamen
vorhanden sind, was eine effektievere Übertragung ermöglicht.

Aus diesen Gründen haben wir uns für eine \ac{REST} Lösung entschieden.

\subsubsection{Design der REST-Schnittstelle}
Bei \ac{REST} gibt es für jede Abfrage eine seperate \ac{URL}. Bei uns liegen
alle URLs im Unterverzeichnis \textrm{/rest}. Insgesamt gibt es 3 URLs:
\begin{itemize}
  \item  \textrm{/rest/info} liefert generelle, statische Informationen über den
  Server, wie z.B. die Server-Version
  \item  \textrm{/rest/pininfo} liefert generelle, statische Informationen über
  die einzelnen Pins, wie z.B. den Namen des Pins
  \item  \textrm{/rest/valus} liefert nur die aktuellen Werte und
  \ac{DDR}-Einträge.
\end{itemize}

Um die dargestellten Werte auf der Webseite zu aktualisieren muss folglich nur
\textrm{/rest/valus} erneut geladen werden. Aus diesem Grund sollte diese
Datei so klein wie möglich gehalten werden um eine optimale
Aktualisierungsgeschwindigkeit zu ermöglichen. \textrm{/rest/info} und  
\textrm{/rest/pininfo} müssen nur einmalig geladen werden, da die enthaltenen
Informationen statisch sind und nie geändert werden. Die Größe dieser Dateien
ist deshalb weniger ausschlaggebend.
%-----------------------------------------------------------------------------------------
\subsection{Implementierung der REST-Schnitstelle auf dem Server}
Für die Implementierung der \ac{REST}-Schnittstelle setzen wir auf dem
besethenden Code aus dem Projekt Radig auf. Typischerweise sind
\ac{REST}-\ac{URL}s virtuelle \ac{URL}s.
Das bedeuted das unter dieser \ac{URL} keine Datei vorhanden ist, sondern diese bei
bedarf dynamisch erzeugt werden. Um die Implementierung möglichst einfach zu
halten haben wir uns dazu entschlossen unter der gegebenen \ac{URL} (also z.B  
\textrm{/rest/values}) eine mit Platzhaltern versehene Datei abzulegen. Das
Vorgehen mit Platzhaltern wurde schon im Projekt Radig verwendet um die
Informationen in den statischen \ac{HTML}-Code einzufügen. Die Platzhalter
werden vor dem Senden der Datei durch zugehörige Werte ersetzt.\\
\\
Das Schema für so einen Platzhalter ist \textrm{\%PINXY} und setzt sich zusammen aus dem
Aufruf \textrm{\%PIN}, dem anzusprechendem Port X = [A,C oder D] und dem Pin Y = [0-7] (z.B.
\textrm{\%PINC1 } ). Mit diesem Platzhalter kann der direkte Pin Ausgelesen
(Hier Port C und Pin 1) und an eine beliebige Stelle im Quellcode platziert werden. Neu
hinzugekommen ist das Ausgeben der Information ob der Konkrete Pin über das
DDRegister als Ein- oder Ausgang definiert ist. Das Schema ist für diesen
Platzhalter ist \textrm{\%DDRXY} und setzt sich zusammen aus dem Aufruf \textrm{\%DDR}, dem
anzusprechendem Port X = [A,C oder D] und dem Pin Y = [0-7] (z.B.
\textrm{\%DDRD1} ). \\
\\
Damit das System für uns möglichst flexibel Arbeitet, haben wir uns dafür
entschieden, diese Dynamischen Angaben in eine Separate Datei auszulagern und
über die REST Schnittstelle abzufragen. Konkreter liegt in dem /Rest
Verzeichnis eine \textrm{values} Datei die die Platzhalter enthält. Zusätzlich
gibt es noch eine \textrm{info} und \textrm{pininfo} Datei, die statische
Informationen zu den Pins und Ports Enthält. \\
\\
Das setzen der Ausgänge wird über einen HTTP-Post Aufruf getätigt. So können die
Pins des entsprechenden Ports gesetzt oder umgeschaltet werden. Die
Informationen, die zum setzen eines Ports benötigt werden setzen sich zusammen
aus einem \textrm{SET} Befehl und dem Aufruf PORTXYZ zum setzen oder dem
\textrm{SET} Befehl und dem Aufruf OUTXYZ. X steht für den anzusprechendem Port
X = [A,C oder D]. Y und Z Stehen für die Einstellung der Pins in hexadezimaler
Schreibweise (00-FF) Abschließend muss das Ende der Schaltanweisung mit
\textrm{SUB} gekennzeichnet werden, da der Webserver auf diese Steuerzeichen
prüft. Ein Beispiel Post zum setzen der Pins C0-C3 auf Ein, der Pins C4-C7 auf
Aus, dem umschalten der Pins D0-D3 als Ausgang und der Pins D4-D7 als Eingang
sieht folgendermaßen aus:
\\

\framebox{SET=PORTC0F\&SET=OUTD0F\&SUB=Senden}
%\begin{itemize}
%	\item Benutzung der Struktur von Radig
%	\item Die Eingägne werden durch Platzhalter (z.B. \%PortC1 ) angegeben. Beim
%			übertragen des Dokuments werden die Platzhalter ersetzt durch passende Werte
%	\item in /rest liegen Dateien values (mit Platzhaltern), pininfo und info
% (jeweils als
%	      statisches Dokument)
%	\item Dateien können von ausßen normal aufgerufen werden und werden ggfs. mit
% realen
%	      Informationen aufgefüllt
%	\item Vorteil: Test außerhalb der Platine möglich, da Werte als String
% gekennzeichnet
%	      sind in JSON, Pins haben dann den Wert \%PortXY anstatt 0 oder 1
%	\item TODO: Wie werden die set URLs implementiert?
%	\item Die Ausgänge werden über die Post übertragung gesetzt.
%\end{itemize}

%-----------------------------------------------------------------------------------------
\subsection{Erweiterung der POST-Parameter}
Um Manipulationen an dem Board vorzunehemen haben wir uns entschlossen das
vorhandenen System zu erweitern. Hierbei werden bei jeder \ac{HTTP}-Anfrage die
POST-Parameter ausgewertet und das Board entsprechend manipuliert.\\
\\
TODO

%-----------------------------------------------------------------------------------------
\subsection{Implementierung der REST-Schnitstelle auf dem Client}
Auf der Webseite müssen die vom Server bereitgestellten \ac{REST}-URLs im
Hintergrund aufgerufen werden können. Hierzu verwendeden wir die bereits in JavaScript
enthaltene \ac{AJAX} Bibliothek. \ac{AJAX} erlaubt JavaScript URLs im
Hintergrund zu laden, ohne das der Nutzer dies bemerkt oder die Seite neu geladen wird. 

\begin{figure}[H]
\lstinputlisting[language=JavaScript]{content/code/ajaxexample.js}
\caption{JavaScript um \textrm{/rest/valus} abzufragen und in ein Objekt zu
parsen}
\label{Ein typischer AJAX-Request}
\end{figure}

Der hier beispielhaft gezeigte Quelltext ermöglicht das laden der \ac{URL}
\textrm{/rest/valus}.
Hierfür wird ein XMLHttpRequest-Objekt verwendet und der geladene Text
(\textrm{x.responseText}) wird mit \textrm{JSON.parse(\ldots)} in ein
JavaScript-Objekt verwandelt.

Sämtliche Server-Kommunikation findet ausschlißlich in \textrm{rest.js} statt.
Hier werden Methoden zum ansprechen des Servers bereit gestellt, welche überall
verwendet werden können ohne Netzwerkcode zu schreiben. Um beliebige Dateien
abzufragen gibt es die Methoden \textrm{loadURL(url)} und
\textrm{loadURLAsync(url, postParams, func)}. \textrm{loadURL(url)} lädt die
Datei synchron und gibt den erhaltenen Text zurück. Das synchrone Laden
bedeutet, das der Aufruf dieser Methode das Programm blockiert, bis die Datei
vollständig geladen wurde. \textrm{loadURLAsync(url, postParams, func)} hingegen
lädt die Datei asynchron im Hintergrund. Sobald das Laden abgeschlossen ist wird
die als Parameter übergebene Funktion \textrm{func} aufgerufen.
\textrm{loadURLAsync(url, postParams, func)} kann außerdem ein Text als
POST-Parameter übergeben werden. Dies ermöglicht das Manipulieren des Boards
über die POST-Parameter, welche vom Board interpretiert werden.

\textrm{rest.js} ruft \textrm{loadURLAsync(\ldots)} in einer festen Frequenz
kontinuirlich auf. Nach jeder aktualisierung wird eine Funktion aufgerufen,
welche über die Methode \textrm{setOnValuesChanged(func)} gesetzt werden kann.

Alle von \textrm{rest.js} zur Verfügugn gestellten Funktionen sowie deren
Verwendung sind der ausführlichen Sourcecode-Dokumentation zu entnehmen.
%\begin{itemize}
%  \item Aufruf der URLs im Hintergrund mit Ajax
%  \item info und pininfo werden zu Beginn einmalig aufgerufen (synchron um zu
%  gewährleisten das die Daten zur Verfügung stehen für andere Initialisierungen)
%  \item values wird mit setTimeout(...) zyklisch assynchron aufgerufen
%  \item synchrones aufrufen von values führt dazu, das sich die Webseite aufhängt, da
%  der (Single-)JavaScript Thread mit dem laden der Daten beschäftigt ist und nicht
%  für andere Aufgaben zur Verfügung steht. Bei einem assynchronen Aufruf werden
%  die Daten von einem anderen Thread im Hintergrund geladen
%  \item JSON-Text wird mit JSON.parse(...) in ein Objekt transformiert
%  \item Informationen werden über entsprechende getter zur Verfügung gestellt
%  \item Funktion onValueChanged wird jedes mal aufgerufen wenn neue Daten zur
%  Verfügung stehen
%  \item onError wird aufgerufen wenn ein Fehler in der Kommunikation aufgetreten ist
%  \item Über setter werden die entsprechenden URLs assynchron aufgerufen und so
% die  Daten an den Server übermittelt
%\end{itemize}

%-----------------------------------------------------------------------------------------

\section{Werkzeuge}

%-----------------------------------------------------------------------------------------
\subsection{Das Atmel Studio}

\subsubsection{Device Programming}

Eine Kernkomponente beim Atmel Studio ist das Device Programming Fenster.
Erreicht werden kann es über "`Tools $\to$  Device Programming"'.
Hier kann die Aktuelle Verbindung mit dem Microcotroller bestimmt werden, es
können die Fuse Bit Einstellung geändert werden oder einzelne Hex
Dateien auf den Mikrocontroller aufgespielt werden.



Übersicht k Gerät Auswählen
Contoller Auswählen
Spannung und Signatur
Fuse Bits
Hexfile Flashen

\begin{figure}[h]
\centering
\includegraphics[width=13cm]{content/pictures/Anleitung/neuerProzessor/AnleitungNeuerProzessor1.png}
\caption{DeviceProgramming}
\label{fig:B3}
\end{figure}

\subsubsection{Projekt Einstellungen}

Taktfrequenz
Programmer
Empfolene Tool Settings

%-----------------------------------------------------------------------------------------
\subsection{AVRDUDE}

Avrdude ist ein Alternatives Werkzeug zum Bearbeiten von Mikrocontrollern. Es
ist im Gegensatz zum Atmel Studio lediglich ein Konsolen Werkzeug ohne
grafische Oberfläche. Mit Avrdude können unter anderem die Fusebits von einem
Mikrocontroller gelesen und gesetzt werden. Weiter ist es möglich eine Sicherung
von dem Aktuellen stand des Mikrocontrollers herzustellen oder ein Hex Datei auf
den Mikrocontroller aufzuspielen. Avrdude unterstützt eine Reihe von ISP Geräten
unter anderem auch den von uns verwendeten Atmel AVRISPmkII. Der Einsatz von
Avrdude war notwendig, da die Fusebits von einem Neuen Microcontroller
Standartmäßig auf den internen Quarz-Kristall gesetzt sind und nicht auf den
Externen Kristall des AVR-NET-IO Boards.
Eine genaue Anleitung gibt es im Kapitel \ref{chap:Benutzerhandbuch}
Benutzerhandbuch.

%-----------------------------------------------------------------------------------------
\subsection{HTML Header Compiler}
\label{chap:hintergrund.HHC}

Zum automatischen umwandeln der Quell-Dateien (html, js, png etc\ldots) haben
wir einen speziellen HTML Header Compiler entwickelt der die Website in eine
für den Webserver verständliche Headerdatei umwandelt. Der Compiler durchsucht
den Eingabe Ordner und sammelt die darin enthaltenen Dateien. Daraus entsteht
dann die für das Projekt benötigte Header Datei, bestehend einem Array von
Buchstaben für jede Datei. Der \ac{HHC} ist den Projektdateien, mit denen diese
Dokumentation ausgeliefert wurde beigelegt, kann aber auch zusammen mit dem
Projekt auf Github \url{https://github.com/doofmars/Embedded-Webserver}
bezogen werden.\\

Eine Beispiel Umwandlung: 

\begin{figure}[H]
\lstinputlisting[language=HTML]{content/code/index.html}
\caption{index.htm}
\label{HHC.input}
\end{figure}

Das eingegebene Verzeichnis, welches die \textrm{index.html} Datei enthält wird
eingelesen und in die folgende Headerdatei umgewandelt.
Abbildung \ref{HHC.input} zeigt eine simple "`Hallo Welt"' Website, die keine
weitere Funktion besitzt. Nachdem die Website mit dem HTML-Header-Compiler
umgewandelt wurde, entsteht die Headerdatei aus Abbildung \ref{HHC.output}. Gut
zu erkenne ist das Array aus Buchstaben in hexadezimaler Schreibweise welches die
index.html Datei widerspiegelt.
Am Ende der Headerdatei ist ein weiteres Array, bestehend aus Schlüssel-Wert
paaren für den Webserver. Hier wird hinterlegt welche Dateien vorhanden sind.
Der letzte Eintrag in diesem Array signalisiert das Ende der Suche und ist die
Bedingung für den Webserver um zur Standardausgabe zu wechseln. Bsp: Eine HTML
Datei wird angefordert die nicht existent ist, resultiert in der Ausgabe von
index.html.

\begin{figure}[H]
\lstinputlisting[language=C]{content/code/webpage.h}
\caption{webpage.h}
\label{HHC.output}
\end{figure}

Anzumerken ist das Standardmäßig der HTML Header Compiler die eingegebenen HTML
und JS Dateien optimiert in die Headerdatei speichert. Dazu wird die
Formatierung für den Zeilenvorschub, Tabulator oder Wagenrücklauf entfernt.
Falls dies für die Entwicklung nicht gewünscht ist, kann man diese Funktion
durch den Parameter \textrm{-n oder -newline} deaktivieren.
Außerdem ist die entstandene Headerdatei nicht zur Bearbeitung gedacht. Die
Bearbeitung erfolgt ausschließlich im Quelltext (z.B. der index.html). Analog
kann man dazu einen beliebigen Compiler einer herkömmlichen Programmiersprache sehen. Dieser
erzeugt aus dem Quelltext eine Binärdatei, diese ist nicht unbedingt
für Menschen lesbar. Bei Änderungen wird der Quelltext angepasst und neu
Compiliert. Da die Headerdatei folglich nicht bearbeitet werden muss haben wir
uns für eine Ausgabe als Array aus Buchstaben in hexadezimaler Schreibweise
entschieden, da es eine einfachere Programmatische Strukturierung erlaubt.

\subsection{AVRISPmkII}


\subsection{AVRJTAGICEmkII}



%Korrekturgelesen: Ann-Sophie Dietrich
\chapter{Benutzerhandbuch} 
\label{chap:Benutzerhandbuch}

\section{WinAVR Projekt zu Atmel Studio}

Um ein bestehendes Projekt in Atmel Studio zu importieren, muss zuerst ein
neues, leeres Projekt angelegt werden.
Dafür wählt man im "`New Project"' Dialog ein neues GCC C Executable Project
aus. Außerdem kann man hier auch den Namen und Speicher-Ort angeben (Abbildung
\ref{import.new}). Nach der Auswahl des Projektes Erscheint eine Tabelle mit den
verschiedenen Mikrocontroller, die unterstützt werden. Hier wählt man den Ziel
Controller aus (Abbildung \ref{new.choose})

\begin{figure}[htp]
\begin{center}
  \includegraphics[width=8.5cm]{content/pictures/Import/1neuesProjekt.png}
  \caption{Atmel Studio, Neues Projekt}
  \label{import.new}
\end{center}
\end{figure}

\begin{figure}[H]
\begin{center}
  \includegraphics[width=8.5cm]{content/pictures/Import/2porzessorAuswahl.png}
  \caption{Atmel Studio, Device Selection}
  \label{new.choose}
\end{center}
\end{figure}

Nachdem das Projekt erstellt wurde, gibt es im Solution Explorer
bereits eine main C Dateien. Diese kann gelöscht werden, da ein
neues bestehendes Projekt importiert wird (Abbildung \ref{new.explorer}).
Im nächsten schritt müssen die Neuen Projektdateien ausgewählt und Importiert
werden. Dies geschieht über einen Rechtsklick in den Solution Explorer
(Abbildung \ref{new.addExisting1}) $\to$ add $\to$ Existing Item. Im geöffnetem
dateiauswahldialog kann das bestehende Projekt ausgewählt werden (Abbildung
\ref{new.addExisting2}).

\begin{figure}[htp]
\begin{center}
  \includegraphics[width=5cm]{content/pictures/Import/3solutionExplorer.png}
  \caption{Atmel Studio, Solution Explorer}
  \label{new.explorer}
\end{center}
\end{figure}

\begin{figure}[htp]
\begin{center}
  \includegraphics[width=7cm]{content/pictures/Import/4addExisting1.png}
  \caption{Atmel Studio, Add Existing Item 1}
  \label{new.addExisting1}
\end{center}
\end{figure}

\begin{figure}[htp]
\begin{center}
  \includegraphics[width=7cm]{content/pictures/Import/5addExisting2.png}
  \caption{Atmel Studio, Add Existing Item 2}
  \label{new.addExisting2}
\end{center}
\end{figure}

\section{Einen Mikrocontroller austauschen}
Für unser Projekt sollen alle notwendigen Programmbestandteile sowie die gesamte
Webseite auf dem Mikrocontroller gespeichert werden. Der beim AVR-Net-IO
mitgelieferte ATmega32 bietet hierfür jedoch nicht ausreichend Speicher.
Wir haben uns deswegen für den aus der gleichen Baureihe stammenden ATmega644P
entschieden, der mit seinen 64KB Programmspeicher den doppelten Speicherplatz
bietet als der kleinere ATmeag32.

Für den Wechsel ist es notwendig, den alten Controller vom Sockel zu entfernen
und den neuen Controller einbauen zu können. Hierfür gibt es spezielle
Werkzeuge, doch wenn man beim Vorgang Vorsicht walten lässt, kann man den
Controller auch mit einem kleinen, möglichst breiten, Schlitzschraubendreher
entfernen. Dazu zuerst den Mikrocontroller vorsichtig mit dem Schraubendreher
als Hebel wie in Abbildung \ref{ausbau1} lösen.

\begin{figure}[H]
\centering
\includegraphics[width=13cm]{content/pictures/Anleitung/tauscheProzessor/1_Hebel.jpg}
\caption{Schraubendreher am Controller}
\label{ausbau1}
\end{figure}

Zum einfachen Lösen kann der Hebel auch von der anderen Seite angesetzt
werden. Anschließend den gelösten Prozessor abziehen.

\begin{figure}[H]
\centering
\includegraphics[width=13cm]{content/pictures/Anleitung/tauscheProzessor/2_Geloest.jpg}
\caption{Der gelöste Mikrocontroller}
\label{ausbau2}
\end{figure}

Nachdem der Mikrocontroller entfernt wurde, hat man einen guten Blick auf den
Sockel (Abb. \ref{ausbau3}).

\begin{figure}[H]
\centering
\includegraphics[width=13cm]{content/pictures/Anleitung/tauscheProzessor/3_Sockel.jpg}
\caption{Der Sockel auf dem AVR-Net-IO}
\label{ausbau3}
\end{figure}

Beim Einbau ist unbedingt darauf zu achten, den neuen Mikrocontroller entsprechend
der D-förmigen Einkerbung in den Sockel zu setzen. (Siehe Abbildung \ref{ausbau4})

\begin{figure}[H]
\centering
\includegraphics[width=13cm]{content/pictures/Anleitung/tauscheProzessor/4_Markierung.jpg}
\caption{Markierung zum Einbau}
\label{ausbau4}
\end{figure}

\section{ISP-Programmer anschließen}

Der Anschluss des AVRISPmkII Programmers erfolgt über einen 6 Poligen Stecker,
allerdings hat das AVR-Net-IO einen 10 Poligen Stecker. Deswegen wurde hierfür
eigens ein Adapter angefertigt, welcher das ISP Signal von den 6 Polen des
Programmers auf das AVR-Net-IO bring.

\begin{figure}[htp]
\begin{center}
  \includegraphics[width=6cm]{content/pictures/Anleitung/ISP-Stecker.png}
  \caption[Schematische Darstellung des ISP Anschlusses]{Schematische Darstellung des ISP Anschlusses (Pin 1 \& 2
  ist auf der Platine markiert)}
  \label{ispanschluss}
\end{center}
\end{figure}

\begin{table}[H]
\centering
\begin{tabular}{|l|l|} \hline
	 \textbf{10-poliger Anschluss} & \textbf{6-poliger Anschluss} \\ \hline
	 1 MOSI & 1 MISO \\ \hline
	 2 VCC & 2 VCC \\ \hline
	 3 - (*) & 3 SCK \\ \hline
	 4,6,8,10 GND & 4 MOSI \\ \hline
	 5 RESET & 5 RESET \\ \hline
	 7 SCK & 6 GND \\ \hline
	 9 MISO &   \\ \hline
\end{tabular}
\caption{Die Pinbelegung für den 6 und 10 poligen Anschluss \cite{mikrocontroller.isp}}
\label{pinbelegung}
\end{table}

\begin{figure}[htp]
\begin{center}
  \includegraphics[width=10cm]{content/pictures/Anleitung/adapter.jpg}
  \caption{Der Adapter}
  \label{adapter}
\end{center}
\end{figure}

Mit dem angefertigtem Adapter kann der Debugger anschließend ganz einfach mit
dem Board verbunden werden. Wenn der Mikrocontroller richtig, wie im Abschnitt
\ref{Chapt:Einrichten} beschrieben, konfiguriert ist, kann der Programmer auch in Atmel
Studio verwendet werden.

\section{Einrichten eines neuen Mikrocontrollers}
\label{Chapt:Einrichten}

Für einen neuen Chip ist es anfangs notwendig die Fuse-Bits richtig zu setzen,
damit der Chip ordnungsgemäß arbeitet.
Dies ist jedoch im AtmelStudio nicht möglich, da es nicht möglich ist die exakte
Geräte-Signatur auszulesen.
Das Problem liegt darin, dass standardmäßig die Fuses auf den internen
Quarz-Kristall gesetzt sind, und nicht auf den externen Kristall des
AVR-NET-IO Boards.
Beim Versuch die Fuse-Bits zu setzen, erscheint im Atmel Studio die
Fehlermeldung aus Abbildung \ref{Einrichten.error}.

Abhilfe schafft hier die alternative Programmiersoftware AVRDUDE, mit ihr ist
es möglich die Fuse-Bits zu ändern. Unter Linux kann dieser einfach über die
Paketquellen installiert werden, für ein Windows Betriebssystem kann eine
ausführbare Kommandozeilen-Anwendung auf der Projekt-Website heruntergeladen
werden \url{http://savannah.nongnu.org/projects/avrdude}. Zusätzlich muss für
Windows noch libusb-win32 (\url{http://sourceforge.net/projects/libusb-win32/})
vorhanden sein, dass der Programmer mit den gewählten Parametern verwendet werden
kann. Eine ausführliche Anleitung gibt es hier:
\url{http://eliaselectronics.com/using-the-avrispmkii-with-avrdude-on-windows/}

\begin{figure}[H]
\centering
\includegraphics[width=13cm]{content/pictures/Anleitung/neuerProzessor/AnleitungNeuerProzessor2_fehler.png}
\caption{DeviceProgramming}
\label{Einrichten.error}
\end{figure}

Die in folgendem Beispiel angezeigten Befehle sind die von uns verwendeten Fuse
Einstellungen. Für eine genauere Beschreibung, wofür die einzelnen Fuse-Bits
verwendet werden, ist der Abschnitt \ref{chap:Fuse} Fusebits im Kapitel 
"`Hardware"'.

Anschließend kann der Mikrocontroller zusammen mit dem AV-Net-IO und AtmelStudio
programmiert werden. Der verwendete Mikrocontroller wird jetzt richtig erkannt,
da es auch keine Probleme mit der Gerätesignatur gibt.

\begin{table}[H]
\begin{tabular}{| p{.24\textwidth} | p{.76\textwidth} |}
\hline
Auslesen Linux:& sudo avrdude -P usb -p m644p -c avrispmkII  -U lfuse:r:-:h -U hfuse:r:-:h -B 22 \\ \hline
Setzen Linux:& sudo avrdude -P usb -p m644p -c avrispmkII -U lfuse:w:0xFF:m -U hfuse:w:0xD6:m -B 22 \\ \hline
Auslesen Windows:& avrdude.exe -p m644p -c avrispmkII -U lfuse:r:-:h -U hfuse:r:-:h -B 22 \\ \hline 
Setzen Windows:& avrdude.exe -p m644p -c avrispmkII -U lfuse:w:0xFF:m -U hfuse:w:0xD6:m -B 22 \\ \hline
\end{tabular}
\caption{Auslesen und setzen von Fuse-Bits des ATmega644P mit AVRDUDE}
\label{ParameterAvrdude1}
\end{table}

\begin{figure}
\centering
\includegraphics[width=13cm]{content/pictures/Anleitung/neuerProzessor/avrOutput.png}
\caption{AVRDUDE Ausgabe}
\end{figure}

\begin{table}
\begin{tabular}{| p{.35\textwidth} | p{.65\textwidth} |}
\hline
-p partno & This is the only option that is mandatory for every invocation of
avrdude.  It specifies the type of the MCU connected to the programmer. These
are read from the config file.  If avrdude does not know about a part that you
have, simply add it to the config file (be sure and submit a patch back to the
author so that it can be incorporated for the next version). \newline
\textbf{m32 $\Rightarrow$ ATmega32} \newline 
\textbf{m644p $\Rightarrow$ ATmega644P} \newline
\textbf{m1284p $\Rightarrow$ ATmega1284P} \\ \hline
-P port & Use port to identify the device to which the programmer is attached. \textbf{usb für den AVRISP MKII}  \\ \hline 
-c programmer-id & \textbf{avrispmkII für den AVRISP MKII} \\ \hline
-U \hbox{memtype:op:filename:filefmt} &  
The \textrm{memtype} field specifies the memory type to operate on.\newline
\textbf{hfuse} The high fuse byte.\newline
\textbf{lfuse} The low fuse byte.\newline
The \textrm{op} field specifies what operation to perform:\newline
\textbf{r} read device memory and write to the specified file\newline
\textbf{w} read data from the specified file and write to the device memory \newline
The filename field indicates the name of the file to read or write.  The format field is optional and contains the format of the file to read or write. \newline
\textbf{Hier die Bytes die gesetzt werden 0xFF bzw 0xD6} \\ \hline
-B bitclock & Specify the bit clock period for the \ac{JTAG} interface or the ISP clock \\ \hline
\end{tabular}
\caption{Auszug AVRDUDE Parameter}
\label{parameterAvrdude2}
\end{table}
\newpage

\section{HTML Header Compiler}
\label{chap:benutzerhandbuch.HHC}

Da der HTML Header Compiler in Java entwickelt wurde, muss für die Verwendung die
Java Laufzeitumgebung ab Version 6 installiert sein. Zum Ausführen des Compiler
muss zuerst mit einer Konsole in den Entsprechenden Ordner navigiert werden.
Anschließend kann mit folgendem Befehl die Datei ausgegeben werden. 
\\

\framebox[1.1\width]{java -jar hhc.jar -in <INPUT FOLDER> -out <OUTPUT FILE>} 
\\

Die Angaben in den spitzen Klammern müssen durch den entsprechenden Pfad und
die entsprechende Datei ausgetauscht werden. Standardmäßig optimiert der
HTML Header Compiler die eingegebenen Dateien, falls dies nicht gewünscht ist
gibt es zusätzlich zu den vorgegebenen Optionen weitere Flags
die gesetzt werden können. Hier alle Parameter im Überblick:

\begin{table}[H]
\begin{tabular}{| p{.24\textwidth} | p{.76\textwidth} |}
\hline
-in, -input & Der Eigabepfad mit allen für die Website benötigten Dateien. z.B. \textrm{-in "Webseite"} \\ \hline 
-out, -output & Die Ausgabe Headerdatei z.B. \textrm{-out "Webserver/webpage.h"} \\ \hline
-v, -verbose &  Gibt die Dateiausgabe auf der Konsole aus.
\\
 \hline 
 -n, -newline & Behält die Formatierung für den Zeilenvorschub, Tabulator
 oder Wagenrücklauf in den HTML und JS Dateien (\textbackslash n and
 \textbackslash r \textbackslash t).
 Benötigt dadurch abhängig von der Website mehr Speicher, ermöglicht aber ein
 einfacheres Debuggen von eingebundenem JavaScript Code.  \\ \hline
\end{tabular}
\caption{Parameter des HTML Header Compiler}
\label{parameterHHC}
\end{table}

Um die Entwicklung zu vereinfachen ist es hilfreich, wenn für den Parameteraufruf
des HTML Header Compilers ein einfaches Shell- oder Batch-Script erstellt wird,
das die Dateien aus dem Ordner für die Website als Headerdatei in den Ordner für den
Webserver schreibt. Falls eine Änderung an der Website vorgenommen wurde, muss
vor dem Programmieren des Mikrocontrollers lediglich der \ac{HHC} ausgeführt
werden.

\begin{figure}[H]
\lstinputlisting[language=sh]{content/code/buildwebpage.sh}
\caption{BuildWebpage.sh für Linux}
\label{output}
\end{figure}

\begin{figure}[H]
\lstinputlisting[language=sh]{content/code/buildwebpage.bat}
\caption{BuildWebpage.bat für Windows}
\label{output}
\end{figure}

\section{Konfiguration des Webservers}

Die Einstellung des Webservers erfolgt über die \textrm{config.h} Datei. In der
\textrm{config.h} Datei, können die verschiedenen Pins der Ports als Ein- oder
Ausgang definiert werden. Dabei gibt es ein paar Eigenheiten zu beachten:
\begin{itemize}
  \item OUTA steht für den A Port, hier ist zu beachten, dass dieser Port die
  Analog zu Digital Wandler beherbergt. Mit aktiviertem Wandler ist es nicht
  möglich, die Pins des Ports funktionierend als Ausgänge zu schalten, da die
  Spannung nicht gehalten wird.
  \item OUTB ist nur mit Vorsicht zu genießen. Hier handelt es sich um den Port,
  der auf dem AVR-Net-IO für die Neztwerkkomunikation genutzt wird. Deswegen
  wird Port B auch nicht standardmäßig definiert.
  \item OUTC dieser Port wird von Pollin standardmäßig für die Ausgänge verwendet
  und ist von uns bereits entsprechend modifiziert. Der gesamte Port wird auf dem
  AVR-NET-IO über den 25Pin D-Sub Stecker geleitet. Wenn der Fuse-Bit für
  \ac{JTAG} geschaltet ist, werden 4 Pins des C Ports für das \ac{JTAG} Interface
  verwendet.
  \item OUTD Liegt auf dem AVR-Net-IO auf dem EXT Anschluss und ist
  für erweiterte Peripherie geplant, so kann hier ein Cardreader oder ein
  Erweiterungsboard angeschlossen werden.
\end{itemize}
Weiter kann die gewünschte IP-Adresse eingestellt werden, unter welcher das
Gerät erreicht werden kann. Wichtig ist hier, dass kein anderes Gerät dieselbe Adresse
im Netzwerk verwendet. Auch kann die Router IP-Adresse und Netzmaske angegeben
werden. Eine weitere wichtige Einstellung ist die verwendete Mac Adresse
des Netzwerkcontrollers. Diese wird über die Variablen MYMAC1-6 definiert.

\section{Debuggen über JTAG}

Der JTAGICEmkII Debugger von Atmel, den wir für unser Projekt gestellt bekommen
haben, unterstützt neben ISP auch \ac{JTAG}. Allerdings werden für den Anschluss von
\ac{JTAG} andere Pins benötigt als für den Anschluss eines ISP-Programmers.

\begin{table}
\begin{longtable}{|l|l|l|p{8.8cm}|}\hline 
Pin & Signal & I/O & Description \\ \hline 
1 & TCK & Output & Test Clock, clock signal from JTAGICE mkII to target JTAG port \\ \hline 
2 & GND & - & Ground \\ \hline 
3 & TDO & Input & Test Data Output, data signal from target JTAG port to JTAGICE mkII \\ \hline 
4 & Vtref & Input & Target reference voltage. Also used to power level converter inputs. \\ \hline 
5 & TMS & Output & Test Mode Select, mode select signal from JTAGICE mkII to target JTAG port \\ \hline 
6 & nSRST & Out/-In-put & Open collector output from adapter to the target system reset. This pin is also an input to the adapter so that the reset initiated on the target application board may be reported to the JTAGICE mkII \\ \hline 
7 & - & - & Not connected \\ \hline 
8 & nTRST & NC(Output) & Not connected, reserved for compatibility with other equipment (JTAG port reset) \\ \hline 
9 & TDI & Output & Test Data Input, data signal from JTAGICE mkII to target JTAG port \\ \hline 
10 & GND & - & Ground \\ \hline 
\end{longtable}
\caption{JTAG Connections \cite{JTAGICEmkII.Quick}}
\label{jtag.Connections}
\end{table}

\begin{figure}[htp]
\begin{center}
  \includegraphics[width=6cm]{content/pictures/jatgPins.png}
  \caption{JTAG Pins}
  \label{jtag.pins}
\end{center}
\end{figure}

Die in Tabelle \ref{jtag.Connections} gezeigten Pins entsprechen der Belegung
der 10 Anschlüssen des JTAGICEmkII. Diese müssen mit den Pins 2-5 des C Ports, dem
Reset Pin, Ground und VCC verbunden werden (Abbildung \ref{jtag.pins}). Die Pins 7 \& 8
des JTAGICEmkII werden nicht mit dem Mikrocontroller verbunden.\\
\\
Damit \ac{JTAG} funktioniert muss der entsprechende \ac{JTAG} Fuse-Bit gesetzt sein.
Dabei ist zu beachten, dass die Pins 2-5 am Port C des Mikrocontroller nicht für
Ein- oder Ausgaben verwendet werden können, sondern bei aktivierten Fuse-Bit
gesperrt sind. Als weiterer Hinweis ist zu beachten, dass die verwendete
Schnittstelle in den "`Projekt-Einstellungen"' umgestellt werden muss. Nach
korrekter Verbindung, die im Device Manager überprüft werden kann, kann das
Programm auf dem Mikrocontroller debuggt werden.

\section{Die Website}

Die Webseite ist in 3 Tabs eingeteilt. 

\subsection{Der Status Tab}
Im Status Tab ist die Übersicht der
einzelnen Pins und Ports des Boards. Fährt der Nutzer mit der Maus über die Pins des
Boards, so erscheint auf der rechten Seite des Browser eine Sidebar mit den
detaillierten Informationen zu diesem Pin. Hier können die Pins oder Ports
als Favoriten gespeichert werden. Genauso können Werte und Skripte der einzelnen
Pins oder Ports geändert werden.\\
Ebenfalls kann der Pin mit einem Mausklick gesetzt werden, oder auf der Sidebar in einer 
Checkbox gesetzt werden. Dies kann jedoch nicht mit jedem Pin oder Port gemacht werden, da 
diese Funktion bei manchen nicht möglich sein darf. Diese Pins sind vom System deaktiviert und 
nicht klickbar.
\subsection{Der Favoriten Tab}
Hier werden tabellarisch die ausgewählten Pins oder Ports angezeigt. Auch hier
können wieder Pin spezifische Modifikationen vorgenommen werden. Löschen der
Favoriten setzt den Pin auf die Standardeinstellungen zurück und ist nicht mehr
in Favoritenliste vorhanden. Pins können wieder über den Status Tab hinzugefügt
werden.

\subsection{Der Einstellungs Tab}
Im Einstellungs-Tab werden die Einstellungen vorgenommen. Hier sind auch die
Informationen über das Board, Version der Webseite und Autoren abrufbar.

Die Import/Export Funktion ermöglicht die Datenbank, in der die
Benutzereinstellungen und Favoriten untergebracht sind, zu exportieren und auf einem anderen System zu
importieren. Die Aktualisierungsrate bestimmt wie schnell sich die Seite
aktualisiert, das ist die einzige Einstellungen die vorgenommen werden kann.

\subsection{Nutzerspezifische Skripte}
Nutzerspezifische Skripte können bei den Pins hinterlegt werden, diese werden
auch ausgeführt. Diese können genauso ein Verbund aus Pins steuern und
manipulieren. Um diese Skripte anpassen zu können werden

\subsubsection{Erstellen von Skripte}
Bei den Pins können auch Skripte hinterlegt werden. Diese Skripte werden in
JavasScript geschrieben nur werden die Tags nicht hinzugefügt. Vorrausgesetzt
werden kleinere JavaScript Kenntnisse, was aber anhand den Beispielen und den
nachfolgenden Skriptkommandos machbar ist.

\begin{table}[H]
\begin{longtable}{|l|p{7cm}|}\hline 
\textbf{Name} & \textbf{Beschreibung}\\\hline\hline
getvalues(string id) & Damit wird der Wert des Pins mit der ID abgerufen\\\hline
setvalues(string id, int value) & Damit wird folgender Wert für den Pin mit der
ID gesetzt\\\hline
getDB(string name, string value) & Damit wird aus der Browserdatenbank
ein Wert(Standardwert) für ein bestimmtes Item (name) gesetzt\\\hline 
putDB(string name, int value) & Hiermit wird in die Datenbank das Item (name)
mit dem value eingespeichert.\\\hline
\end{longtable}
\caption{Einige wichtige Skriptkommandos}
\end{table}

\newpage
\subsubsection{Beispielskripte}
Die folgenden Beispielskripte können verwendet werden, müssen eventuell noch auf
die passenden Pins und Ports angepasst werden. 

\subsubsection{Mehrer Lichter blinken}

Hier werden mehrere LED an die Pins D2-D7 angeschlossen\
\begin{figure}[H]
\lstinputlisting[language=javascript]{content/code/skript_example1.js}
\caption{Beispielskript: Mehrer Lichter blinken lassen}
\label{output}
\end{figure}

Durch den Aufruf dieses Skriptes wird zuerst überprüft ob D2 auf 1
gesetzt ist, ist dies nicht der Fall wird D2 auf 1 gesetzt. Von D3-D7
werden Die Werte abwechselnd 0 und 1 gesetzt. Da dieses Skript dauernd
erneut aufgerufen wird ist somit beim nächsten Aufruf D2 bei 1 und der
Wert wird folglich auf 0 gesetzt. Alle anderen Werte werden auch
geändert. Somit entsteht ein wechseln der Lichter.\newline

\subsubsection{Umwandeln in digitale Werte}
Hier werden auch wieder 7 LED verwendet, und ein Sensor angeschlossen
an einen analogen Ports.
\begin{figure}[H]
\lstinputlisting[language=javascript]{content/code/skript_example2.js}
\caption{Beispielskript: Werte werden in digitale Werte umgewandelt}
\label{output}
\end{figure}
Je nachdem wie sich der Wert von dem Sensor ändert, änderen sich auch die
digitalen Ausgänge zu den LED, ist der gelierferte Wert oberhalb von
869, so wird Anschluss D2 auf 1 gesetzt. Leuchte an D2 leuchtet somit. Da dann
aber alle Abfragen wahr sind werden alle Lichter angeschaltet. sollte dann der
Wert sich unterhalb von 869 befinden dann wird D2 ausgeschalten. Somit ändern
sich immer die Leuchten sobald die neuen Werte überprüft werden.

\subsubsection{Lichterlauf}
Bnötigt werden auch hier wieder 7 LED. Angeschlossen an D2 - D7.\\
Mittels diesem Skript werden forlaufend die Lichter nacheinander an und wieder
abgeschalten. Somit entsteht ein Lauf des Aufleuchtens.\newline

\begin{figure}[H]
\lstinputlisting[language=javascript]{content/code/skript_example3.js}
\caption{Beispielskript: Lichterlauf}
\label{output}
\end{figure}
Zur Erklärung des Skriptes. In einem Case der Switch-Anweisung wird das
vorherige LED ausgeschalten, die nächste LED angeschalten und den Counter
erhöht. Beim nächsten Durchlauf wiederholt sich das immer wieder.

\subsubsection{Countdown mit Button}
Für diese Skript wird die 7 Segment Anzeige und ein Button benötigt.\\
Diese Skript lässt die 7 Segment Anzeige nacheinander durchiterieren, sobald
ein button eine bestimmte Zeit gedrückt wird.

\lstinputlisting[language=javascript]{content/code/skript_example4.js}

Sobald der Button angeschlossen an A0 gedrückt wird, wird ein Timer ausgelöst.
ist dieser Timer bei 8 und der Button noch gedrückt, wird bei den Anschlüssen
für die Anzeige durchiteriert. Somit ändert sich nur die Anzeige sollte der
Button gedrückt werden. 

\chapter{Rückblick}

\section{Soll/Ist-Vergleich}

\section{Verworfene Varianten}

\section{Eigenbewertung}

\subsection*{Jan-Henrik Preuß}

\subsection*{Christian Würthner}

\subsection*{Ann-Sophie Dietrich}

\subsection*{Marcel Schlipf}
\chapter{Ausblick}

Brauchen wir den Ausblick?
\chapter{Fazit}

Die Welt der Mikrocontroller steckt voller Möglichkeiten ist aber auch mit
einigen Schwierigkeiten behaftet. Anders als beim arbeiten mit Computern bei
denen der Speicherplatz für einfache Programme schier unbegrenzt ist kommt es bei
den Mikrocontroller auf jedes Byte an. So bestand in unserem Projekt nicht nur
die Schwierigkeit darin den Server mit weiteren Funktionen auszustatten sondern
auch bei der Programmierung möglichst auf Effizienz zu achten und den
bestehenden Webserver von nicht benötigten Funktionen zu befreien. Als eine
weitere Herausforderung bei Mikrocontrollern kommt noch die ganze elektronische
Seite hinzu. Als Informatiker haben wir durch das Studium kaum Berührung mit
diesem Thema gehabt und mussten uns vielerorts in die Themaktik einarbeiten.
Doch hat sich das Projekt als handhabbarer erwiesen als anfangs gedacht. Die
Hauptaufgaben bestanden hier im beschaffen von Bauteilen, dem
erstellen von Platinen zum Testen der Funktionen oder dem Programmieren und
Debuggen des Mikrocontrollers.


\chapter{Anhang}

\begin{figure}[H]
  \includegraphics[width=14cm]{content/pictures/AVR-NET-IO_schaltplan.png}
  \caption{Schaltplan AVR-Net-IO Board}
  \label{anh:schaltplan}
\end{figure}

\begin{figure}[H]
  \includegraphics[width=16cm]{content/pictures/Anforderungen.pdf}
  \caption{Anforderungsdefinition zu Beginn des Projektes}
  \label{anh:schaltplan}
\end{figure}


% Schalgwortverzeichnis (Index)
%\printindex

% Literaturverzeichnis
\singlespacing
\bibliographystyle{alphadin}
\bibliography{bibtex}

% Eidesstattliche Erklärung
%\include{content/affirmation}

\appendix
% Hier können Anhaenge angefuegt werden

\end{document}      